Season�s Greetings, 	\hfill						December 2009
\vspace{.5in}

2009 has been a year!  We had the opportunity to share the greatest joy and at the same time, we were embroiled in unbelievable worry and fear.  But, the best part is that our Christmas wishes were granted.

Nasr had a very busy year. It started out with a gold embossed invitation to President Barack Obama�s inauguration.  Yes, we have that one framed.  No, we did not go because it there would be too many people and it was way, way, way, too cold.  Nasr is very proud of his warrior son. While Adam was training at Camp Pendleton, someone took a picture of him in full body armor holding his M16 at the ready with his ammunition draped over his shoulder.  Adam looks quite intimidating in this photo.  To Nasr, it is a treasure that he wants to flaunt.  Therefore, Nasr took this picture and made it into a huge poster.  Once he got the picture home he proudly carried it to the living room and wanted to put it over our door.  It took a lot of sweet talking, but I finally convinced Nasr that the photo of the modern, tough Marine did not really fit well with our white, ornate, Victorian furniture and the French tapestries.   However, it was perched in the front bay window for a few weeks.  (I think he really wanted to intimidate any would-be burglars�and show off his pride in his son.)  Now, thank goodness, it resides proudly above Nasr�s office door.  I really wonder what his UCLA students are thinking.  It is my guess that they know they better study and  get good grades�or else!  Nasr loves to do projects around the house.  Well, since we�ve been in our house for over 22 years, we sort have run out of projects to do here.  But�.Amira and Matthew bought a new house this summer.  Oh, my goodness, Nasr was in seventh heaven.  This was the perfect opportunity to help his children AND to work on house projects.  Virginia (that is the state in which Amira and Matthew reside) has an abundance of beautiful green trees.   It is my guess that there must be 100 trees for every person in Virginia�maybe more.  In their new yard tree branches hung ominously over the house and the bushes and flower beds were grossly over grown.  Nasr was salivating at the chance to help.  So, he hired a crew of tree trimmers and gardeners and in less than a week their house visibly brightened from the inside and now passersby�s can see that that they have a lovely home.  It was a bit of a nail bitter for Amira and Matt because they were not sure if Nasr was going to tear down every limb and leaf.  But, whew, he stopped in the nick of time and created a beauty.

Virginia is a grandmother.  Ok, that was painful to write.  Having a grandchild gives you the most rewarding, happy, fulfilling moments that you can imagine.  When you look at your grandchild you see your own children and you feel those same wonderful feelings that you felt when you held your own child.  But, this is even more intense.  The only issue I have is that now I am called Grandma. When I hear that word,  I picture old grey haired ladies wobbling down the street.   I don�t see myself that way.  The scary part is that my mother was my age when she became a grandmother and I thought she was really old.  Now, I realize, my mother was a very young grandmother�just like me.  Even though there are days when I feel as old and achy as Granny, make sure you call me �Teta.�  Also, I intend to be very involved as a grandma.  Now I just have to find a way to live in Washington D.C. or find a way to bring them back to LA.  I know.  Both are almost impossible, but I am asking Nasr for four round trip tickets to D.C.  for Christmas.  (Maybe I will sneak in a few more on my own bank account, too.)  There was another first for me this year.  I know many people see me as a bit of a character�ok my siblings think I�m a little eccentric. (They might say crazy, but I think that is a little harsh.)   But, this year I was actually a character in a play.  California State University, Long Beach wrote and produced a play entitled,  �And the War Came.�  This play was a collection of vignettes�or little stories�of people�s experiences with the war in Iraq.  One of the professors knew that Adam had been in Iraq so they asked to interview me.  It was a powerful play�pretty difficult to watch at times�but it was enlightening to see so many aspects of  the war and how it affected different people.  They even entitled one of the vignettes, �Virginia.� Ok, the character/actor was 25 years younger than me and a �bit� thinner, but she vividly expressed my thoughts and the feelings of many military families.  I really hope that more people could have seen it.

Amira and Matthew have multiplied and have welcomed their beautiful dark haired, brown eyed ,  rosy, chubby, cheeked, beauty into their home.  Siena Malika Koerner was born on August 12, 2009.   Both Amira and Matthew are great parents.  They have the patience, love and knowledge to take care of their daughter.   These two tried to do it all this year.  They moved into their new home in July and brought Siena home in mid August.  Moving into a home is a huge undertaking without the added burden and worry of a pregnancy.  I have to say we raised a hard working daughter.  She went into work in the morning and went to the hospital that night to deliver.  She is so responsible that she attends an �important� meeting while she is having contractions.  Amira is sleep deprived and wondering how old Siena will be when she sleeps through the night and Amira will be able to sleep peacefully.  I hate to break her bubble, but my children are much older and I still have trouble sleeping peacefully at times.   Gidu (Grandpa) and Teta had a chance to visit Siena�ok, we also visited Amira and Matthew�in  November. Siena had changed so much since we saw her as a newborn.  Siena babbles non-stop. (I have no idea where gets that from.) She and I have great conversations and I haven�t decided if she is going to be a singer or a politician.  I am inclined to think it may be the latter because her double talk doesn�t make a lot of sense.  Matthew is making sure that Siena will be educated in great literature.  While Amira feeds her, Matthew reads Melville�s classic, �Moby Dick.�  He will tell you it is for Amira and himself, but I know he has great plans for his beautiful daughter.  Oh, just to let you know, Gidu is planning on teaching her calculus in a few months.  He figures he needs to start early.  

Adam is our hero.  He does not consider himself a hero�as most of the Marines believe.  These young men put their lives on the line every day so we can enjoy our freedom and our safety.   Until I had a child in the military, I never knew how much military families give to this country.  As I am writing this letter, Adam is returning from Afghanistan.   I must admit he had a chance to see the world. On the way there, he stopped in Ireland, and Romania. The first stop out of Afghanistan is Kurgistan. While he was waiting to leave Kurgistan, he stood in the foothills of the Himalayan mountains gazing at the huge snowcapped peaks. But, his stay in Afghanistan was not a cool stroll in beautiful nature.  He was stationed at a Forward Operating Base in theintense heat of the Helmand Province.  By the time he left, he had built walls and structures for the troops that were relieving him, but his stay was fraught with 130 degree temperatures, sleeping in the open, unsanitary conditions and firing his Howitzer daily.  Marines live with absolute terror mixed with unbelievable boredom. He did appreciate the small computer we sent that was filled with movies. (No, there was no internet and he could only use a satellite phone every few weeks.)   Of course he has not been able to tell us what it was like, other than they lived in what looked and felt like moon dust.  Every item they used or received was covered in dust.  They lived on MRE bars�yes tasteless bars that give you necessary nutrients, but no flavor.  He was living in these conditions for 7 months.  If all goes as expected, Adam will be home tomorrow�yes there were several delays and he will be finishing his engineering degree and he will be forever finished with the deployment.  We have been babysitting Adam�s dog, Max. (We really would like a real grandchild�I just thought I would let Adam know.) Max is a Toy Fox Terrier.  He stands about  8 inches tall so you realize that he looks at most people�s ankles.  So, toes and feet are very accessible for him.  I know it is a dream of many of you that you would have so much adoration from someone that they kiss your feet.  Let me tell you.  Max loves to kiss and lick feet and toes.  It tickles and it is almost impossible to escape from Max�s determination.  Next time you wish someone will kiss your feet, just let me know and I�ll send Max and you will definitely change your mind.

Jasmine has made the big move out of the house.  She was accepted to University of California, San Diego.  Her major is human biology and at this point is planning on going to medical school. (Anyone want to lend some poor parents some tuition money?) Jasmine is now living with two other young ladies.  There are words coming out of Jazz�s mouth that I never thought would happen.  She cannot believe that her roommates are so messy.  She complains that she has to clean up after them.  Why can�t they keep the kitchen clean? She wonders.  Nasr and I don�t know if we should laugh or cry because she would never have thought of cleaning up after herself when she lived at home. She is learning how to cook, too.  I wonder why she never had the urge to learn at home?  Oh, Jazz has to take very high level calculus.  She didn�t have much of a background in it, so her father spends many an evening with her tutoring calculus.  All I can say is thank God for the computer, the camera and Skype.  Nasr went out and got a white board on an easel and focuses the camera on the easel and proceeds to teach her all she needs to know in calculus.  I am telling Nasr that he should go on the internet and make a lot of money tutoring students. (Hey, that might pay for Jasmine�s medical school.   Jasmine took her cat to her apartment.  Her roommate brought her dog for the weekend and somehow, magically,  the cat ended up with fleas.  She gave the cat endless flea bath, complained, more flea medicine, complained loudly to mom and dad and then finally brought the cat home for us to deal with.  No, by now, the fleas are taken care of�thanks to mom and dad.  Being independent is not what many people think it is cracked up to be.  
In case you are wondering what two wishes were granted, I will tell you.  I wished for a  a healthy grandchild and for Adam to return from war in one piece.  As every year I hope your year has been filled with adventure and blissful times and let us hope that all our young men fighting for us come home to celebrate a joyous New Year.
