%\documentstyle[11pt]{article}
%\setlength{\topmargin}{-.75in} \setlength{\oddsidemargin}{-.3in} \setlength{\evensidemargin}{-.3in} \setlength{\textwidth}{7in}
%\setlength{\textheight}{9.75in} \setlength{\parindent}{0.2in} \setlength{\parskip}{0.1in}

%\begin{document}

Happy Holidays\hfill December 1996

I'm trying to get an early start on my letter this year, but early is a relative concept. By the time you actually get the letter, it may be
January or February. I have the music turned up loud so it will put me into the proper mood, but once I start writing and thinking about our
friends, I don't need any outside stimulus to feel the holiday spirit.

Nasr has been very busy this year. He is expanding beyond theory into the challenging field of experimental engineering. Bulky equipment fills
his laboratory and it looks a little like Dr. Frankenstein's lab. There are thousands of wires and enormous power producers. He says he's
working on the "plasma processing of materials to produce surface coatings for many applications." (I would like a coating of mink to cover my
surface, but Nasr says he can only make diamond coatings. I guess I wouldn't mind being the first on the block to have a diamond coat.) He is
travelling to Japan in December and he's presenting pearls of wisdom to the academic community and then he's bringing pearls on a necklace,
bracelet and ring to me. (I can dream, can't I?) Nasr's parents and sister visited us for two wonderful months. We did a lot of sightseeing,
shopping and visiting. Like many people in Nasr's age group, he has realized the importance of exercise. We did quite a bit of bicycling this
summer, and he did purchase an immense weight machine and it is still waiting for us. Now ladies, don't you agree that your husbands are your
Knights in Shining Armor. After 20 years of marriage, Nasr is still my Knight, even though his armor might be a little tight and tarnished.
Knights are supposed to slay dragons, and unfortunately for the Knights of today, there are no longer dragons. But, I believe that the modern
day dragons are those nasty ants that march into the house when it is too hot or wet outside. So, in order to maintain his Knightlyness, I feel
he should slay the ants. I'll play Maiden in Distress, (I have mastered that role.) and he can slay the dragon ants. Don't you agree?

Virginia is still coordinator at Gault Street School. When people ask me what I do, I usually tell them, "I do whatever needs to be done." I'm
still taking tests for Spanish and busy playing Mom. I'm trying to knit an Afghan, and it's already been claimed by all the children. Little do
they know, that it will most likely take me 15 years to finish it.

Amira (oops, I mean Ami) at 17 (in one month) is busy getting ready to go to college next year. She is more ready for it than we are. This young
lady is being constantly tested. She has to take dozens of alphabet soup exams, fill out college applications, write college essays, study for
her AP Calculus class, AP Chemistry, English, Physics, take piano lessons and she still has time to be captain of her high school soccer team.
Ami also has transportation. She thinks she is the only person in the world who has to drive a car that does not have enough power to get out of
the San Fernando Valley and is put together with duct tape (duct tape holds the light in place). According to Mom and Dad, she should consider
herself very lucky to have any sort of wheels that will get her from point A to point B. She, however, looks at it from a very different
perspective. Her car is ugly and she should drive a beautiful brand new car. Mom and Dad think we should pay for college first. What do you
think? I must admit she has been very helpful in driving the other children to wherever they need to go. Mom's spoiled and doesn't know what she
is going to do next year when she has to go back to driving.

Adam is 14 and has grown. He's grown and grown and grown. He's taller than Mom. He's taller than Dad.   But, he's not the tallest on his
basketball team, he's second tallest. His best friend is the only one taller than him.   Basketball has become his forte. His freshman
basketball team won the High School League Championships. The girls were teasing him about having his name announced over the high school public
address system. You could tell how badly it bothered him whenever you looked up at his face and the sun glinted off his pearly whites.
Unfortunately, Adam's vocabulary has become quite limited now that he is in high school. When Nasr asked Adam how was his day, Adam's response
was a mumbled "Idno." All his words are swished together and unintelligible. Nasr is sure he's speaking a foreign language. The other day, after
Adam spoke to him, Nasr looked up at me with a puzzled look on his face and asked, "Can teenagers understand each other?" Guitar is the
instrument of Adam's choice. He is taking weekly guitar lessons and is doing quite well. We bought him a guitar and amplifier in October. He
agreed that both the guitar and amplifier would be his Christmas present, but that was way, way back in October. Now in November he needs some
other \$200.00 present. Does anyone else see something wrong with this picture?   Adam is also into computers.   He cannot spend time with his
friends without having the computer on.   Adam will invite a friend over and then Adam will sit on the computer downstairs and his friend will
sit at the computer upstairs. They play games against each other through the modem and talk to each other on the other telephone line. Yes, both
lines are tied up. Oh, and a word to the warning. Do not pick up the telephone line that is hooked to the modem as it immediately cuts off the
modem. Once that happens you will hear Adam screaming through the house, "You ruined it!   Who picked up the phone?   I'm going to kill you." I
tried to explain to Adam that it would be much easier to sit and talk to his friend face to face and then we could use one of the phones, but I
didn't get very far. Is this what technology brings?

Jasmine is ten and in the fifth grade.   She loves horses, loves art and hates to pick up after herself. She is attending the Conservatory of
Fine Arts at California State University Los Angeles. This year is much easier because several families are car pooling and I have a couple of
Saturdays free every month. Jasmine is a true artist. They can make everything so beautiful and so messy at the same time.   Jasmine moved to a
room that is well hidden from visitors. This is the only room that looks like "Twister" visits twice a day. But, a hotel room is an entirely
different story for Jazz. The minute we arrive, she has to put all her clothes on hangers. (Of course, that leaves no hangers for the rest of
the family.)   She even hangs up her dry beach towels. She walks around the room meticulously organizing everything on the tables, the bags, and
the contents of the refrigerator. Then, she stands in front of me, folds her arms and says "Can't you even say thank you for cleaning?"   Where
did that come from?   She can't pick up one sock in her room, but she plays maid in the hotel.   I've decided that I'm putting a sign outside
the door that reads "Hotel Ghoniem" maybe this way, Jazz will clean up her room.   I know, I love to dream.


The year has come to an end, but I'm ready for new beginnings.   Our home is once again a home, not a construction site.    I think I'll really
have the Hotel Ghoniem sign made, not just for Jazz, but so all our wonderful friends can visit.   We pray that the new year will bring joy and
happiness to all our friends and family. .

%\end{document}
