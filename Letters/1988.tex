%\documentstyle[11pt]{article}
%\setlength{\topmargin}{-.75in} \setlength{\oddsidemargin}{-.3in} \setlength{\evensidemargin}{-.3in} \setlength{\textwidth}{7in}
%\setlength{\textheight}{9.75in} \setlength{\parindent}{0.2in} \setlength{\parskip}{0.1in}

%\begin{document}

SEASONS GREETINGS\hfill December 1988\\

I can't believe another year has come and gone. This year has been so busy for us that we're still a little unsettled. Well, I better say we're
a lot unsettled. For those of you who haven't heard, we've moved! My hat goes off to those who can sell their house every five years. For me,
AAGHH!!! I'm never going to move again. I'm going to grow roots in this house and you'll have to dig me out to make me move. We've all seen the
commercials about how easy it is to move and how enjoyable an experience it is, well, I have news for you. They lie. I won't go into the gory
details because I'm not one to write horror stories. The only good part is that it is a horror story with a happy ending-a real rarity these
days. Our new house is big, 5 bedrooms, and we have a half an acre of land. (To some, that may not sound like a lot, but in Los Angeles, that's
the Ponderosa!) Nasr has a wonderful time walking his estate and a miserable time dealing with the contractors. It will look gorgeous and
everyone will have to come and visit us, but don't act surprised when you trip over my roots. Our new address is:
\begin{center}
11809 Andasol Avenue\\
Granada Hills,   Ca.   91344 \\
Tel.   (818) 366-4846
\end{center}

NASR had a magical turning point in his life this year. He turned 40! For some 40 may come as a time to regret and fuss, but Nasr really took it
quite well. Even though his back is not that of a twenty year old, he's still lifting weights every morning and trying desperately to look like
Arnold Schwarzenegger. Like I said, he took 40 quite well. I wonder who'll he'll try to look like at 50? He's been very busy at the university
and swamped with trying to get all the landscaping done. Our half acre lot now is one giant desert (you know, like the Sahara in Egypt). Nasr is
struggling to turn it into an oasis of beauty. I think he'll succeed, but right now the task seems formidable. Hopefully, in next year's letter
I'll be able to tell you how gorgeous it looks.

VIRGINIA looks at 1988 as a year of success. I even succeeded in keeping a New Year's Resolution. I was resolute to lose 15 pounds. I lost that,
plus 15 more! I think this is the first year in my 29 plus years that I actually kept a New Year's Resolution. But there is more, since I lost
the weight through Weight Watchers, they hired me to be a meeting leader. Yes, I now am a working mother. No, let me clarify that. I'm a mother
getting paid for working outside the home. Every mother is a working mother, only some of us get paid for it. Now that I've lost so much it
seems I have endless energy. Well, not endless, but more. Anyway, on top of being Mom (meaning chief cook, bottle washer, chauffeur, maid and
referee), I'm also going back to school to get my teaching credentials. So I'm a studying, getting paid for my work Mom. My writing. I'm still
writing, but it's off to the side while I finish my schooling. Teaching is great. I'll have my summers off and do my writing then. I have it all
planned out. Ask me next year how my plans are going. If I answer in a mumbled, incoherent manner, you'll know everything is under control.

AMIRA   is   eight,   will   be nine   in   January,   but   I   feel   I   have   a   teenager   in   the house.         Isn't    the    loud
music,    funny    moods,    strange    fashions    and    over    all weirdness    mixed    in    with    moments    of    concern    and
helpfulness,    all    traits    of    a full fledged teenager? If you tell me its going to get worse I think I'll cry. I think God should
supply every woman who is contemplating getting pregnant with a yearly breakdown of what to expect from the kids. I don't mean a watered down
nice version, I mean the real horror story. I know Mom always excused my weirdness as phases to my sisters, but good grief. There's no way I
could have acted like that, was there Mom? Please tell me I can blame it on someone other than my own genes. Amira is active in ballet and plays
piano beautifully, she's doing well in school and loves the new neighborhood because it is filled with kids and most of them are her age.

ADAM is a very strong six year old. Yes, he is physically strong, but he is also a very strong willed child. I'm trying to use a nice euphemism
for stubborn, but it just doesn't truly express how it really is. Adam is doing wonderfully in school. He is reading sixth grade novels from the
library even though he is only in first grade. He was taking piano lessons, but stopped because Mom got tired of "convincing" him to practice.
He does very well in karate. He is a very agile, strong and coordinated child. If he gets much better in karate I will have a hard time
controlling my strong willed child. He and Jasmine tolerate each other. Only occasionally do I see him irritating her. He and Amira now are
competing for some negative attention. They are beating each other, screaming at each other and having a wonderful time doing it. Mom, however,
is on the verge of a nervous breakdown. I finally realized why women go back to work when the children get older. No matter how chaotic a job
you undertake, it's got to be a thousand times easier than being a Mom.

JASMINE, at two, is Nasr's shadow. From the time Nasr walks in the door, until he goes to sleep Jasmine is with him. The other kids are so busy
with their friends that Nasr felt ignored for a while. But now Jasmine is completely filling his time. She just recently gave up the bottle,so
now we have a major war every evening to get her to sleep. More often than not, I'll find Jasmine in Nasr's arms and he's fast asleep and she's
busy playing with his mustache and nose. Jasmine is trying to do everything that her big sister does. She saw Amira in her ballet costume, so
she INSISTED that she wear one of Amira's old tutu's. Now Jasmine is forever wearing a bright yellow tutu over her winter blue jeans and long
sleeve shirts. She eats, breathes and sleeps in that tutu. All hec will come down if we try to remove it from her body. (And believe me, that
presents a problem when I'm trying to potty train her.) I'm sure she will be a clothes hound when she grows up. Most children love to hold teddy
bears when they sleep, but Jasmine is just a little different.   She wants to hug Amira's bright, shiny, black patent leather shoes when she
sleeps. I hope a psychiatrist doesn't read this letter. Who knows what he'd think?

All joking and kidding aside, this year has been a good one for all of us. We spent a wonderful time visiting family in both New Orleans and in
Wisconsin this summer. My parents celebrated their 50th anniversary. Now that is something to work towards. Look, they did it and they had six
kids! Doggone, now I have no excuses for not doing as well. This year we find ourselves in good health and happiness and we hope that this
letter will find all of you healthy and happy.



%\end{document}
