%\documentstyle[11pt]{article}
%\setlength{\topmargin}{-.75in} \setlength{\oddsidemargin}{-.3in} \setlength{\evensidemargin}{-.3in} \setlength{\textwidth}{7in}
%\setlength{\textheight}{9.75in} \setlength{\parindent}{0.2in} \setlength{\parskip}{0.1in}

%\begin{document}

Happy Holidays! \hfill December 1989\\


I'd sing the song and send it on a tape, but my children insist that I have such a screeching voice that everyone would think I was sending out
a Halloween tape. Well, maybe my voice has gotten a little screechy now, but in BC (Before Children) I used to sing in choirs and I'd sing
lovely lilting melodies. How do you think my voice could have changed so much in ten years???? Even if I can't sing it, I want to wish all of
you the happiest of holidays. 1989 has brought a lot of changes in our life and lots of fun and laughter, too.

Nasr has had a busy year. Yes, the son of gun got to travel again. This year he went to the French Island of Corsica, Egypt and Denmark. This is
all business!?! Let me tell you about his business. On the sunny warm island of Corsica overlooking the Mediterranean is where they hold their
conferences. They confer from 9am to noon. Then these poor souls are forced to lie on the beaches absorbing the local color and attractions.
Now, during this sacred holiday time I will not say what these beaches are lacking, but I'm sure with a bit of imagination you can all see what
he saw. Nasr suffered two weeks of this torture and then went on to spend a wonderful week with his parents and then another week in Denmark.
Oh, yes, Nasr does work. He spends long days at the University and this year he is vice-chairman of the department. Oops, I forgot, he's also
been to Japan two or three times this year. I can't be too accurate because it's hard to keep track of where he is and when.

Virginia survived the month that Nasr travelled. I won't say there weren't any casualties, only my sanity, and my lilting voice, but I pulled
through and I'm able to continue until his next trip. As a mother, I worked very hard. Now, as a working mother I work very, very, very hard. I
heard the horror stories of working mothers, but I didn't think they were real. Believe me, any story you've heard has to have been watered down
because the real picture is too blood curdling and too much to bear. I still have all my mother duties such as helping with homework,
chauffeuring, cooking, chauffeuring, cleaning house, chauffeuring, laundry, chauffeuring, bills and chauffeuring. But now I have all the duties
of a person. I'm going to school and taking two classes for my education credential, and I'm studying Spanish. I still work part time for Weight
Watchers and I'm a substitute teacher in the Los Angeles Unified School District. I love teaching, but being a substitute has its ups and downs.
Remember the days you had a substitute? How did you behave? Believe me, what you did "ain't nuthin" compared to what these kids can do if they
want to. I sub in the Spanish bilingual classes and these classes can be wonderful or if they want, they can completely shut you out just by
speaking very swiftly in Spanish. (Now you know why I'm studying Spanish.) At least with teaching, I'm home in the afternoons when the children
are home. 1989 has been packed full of things for Virginia, and to tell the truth I'm a bit leery to open the doors on 1990. The 90's are
supposed to be a busy time for everyone. If its going to get busy I think they better reserve a room for me with padded walls.

Amira will be ten in January. Or maybe I should call it ten-teen. If I'm getting preview of the years to come, turn it off. I like surprises. I
can wait. The only problem is that she can't wait. Amira is doing very well in school and she is still taking piano lessons and doing
wonderfully at her recitals. However, practice time seems to be cutting into her time with her friends so we do have a few minor skirmishes over
that. Amira's newest interest is cheerleading. She and a friend started a cheerleading squad at her school and they spent a lot of time working
on it. The only problem is that whenever you try to talk to Ami (she likes the shortened version) , she is jumping around and swinging her arms
in some sort of cheer. I feel my head is bobbing up and down just trying to keep up with her. Even at the table she's talking, eating, twisting,
jumping and rah rahing all through dinner. I'd like to think she's cheering on my food, but most of the dinners I make now are not up to her
taste. I usually hear, "Ah Mom, that's gross!" Rah! Rah!

Adam is seven and will be very successful in life. Anyone who is that strong willed will get his way if it kills him, or breaks the eardrum of
his opponent. We have a future karate champion in the house. Already Adam is winning trophies and medals in the tournaments. I'm not sure who is
prouder of his accomplishments, Adam or Dad. Adam is doing very well in school and is reading just about anything he can get his hands on. Radar
and Adam are inseparable. Well, almost. Adam loves the cat and you could say he loves her to death. If he wants to hold the cat he is GOING TO
HOLD THE CAT and it is unimportant that Radar doesn't want to be held. We are trying to convince him that when Radar growls she wants to be let
go, but Adam doesn't seem to understand. Even the scars across his face haven't convinced him. Now, you'd think that Radar would be smart and
try to avoid the THE TRAPPER as we call Adam, but the minute he wakes up she is following him around from room to room. Adam has a kind heart
for animals. Due to Jasmine's generosity with the fish food, (she dumped the entire can of food into the tank) we lost our fish. There was one
fish floundering and Adam rescued it and the three children tried to give it mouth to mouth resuscitation. (They used a straw). Their idea made
sense, but fish need oxygen in the water and they were blowing carbon dioxide into it. But, remember, it's the sentiment that counts.

Jasmine, at three, has a flair for fashion. She insists on wearing certain outfits and believe me she refuses very loudly if in her opinion a
particular outfit is "UGWI"! Her favorite clothes are Amira's summer dresses that are so long that she trips over the hem. She has a preference
for ruffles and going barefoot. The neighbors call her the Gypsy Queen because they used to see her running up and down the sidewalk in her long
dress, long blonde curls and barefoot. In another of her fits of fashion, Jasmine decided to cut her hair. She really knows what is in fashion
because she gave herself a perfect mohawk. The only thing she forgot was the gel. She now has very short hair, but it fits her personality.
Because her hair is so short, I try to decorate it with pretty feminine bows and headbands. I bought a very pretty headband with a pink bow on
the top. Jasmine has no desire to wear the bow on top of her head, she puts it on her forehead and she looks like a doctor wearing that mirror
type thing on his forehead. Maybe this is a preview of her future profession, or maybe it just tells us how off the wall our daughter is.

1989 was wonderful for us and we pray it was as good for all of you and that 1990 will be filled with happiness and joy for all.



%\end{document}
