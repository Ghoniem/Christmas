%\documentstyle[11pt]{article}
%\setlength{\topmargin}{-.75in} \setlength{\oddsidemargin}{-.3in} \setlength{\evensidemargin}{-.3in} \setlength{\textwidth}{7in}
%\setlength{\textheight}{9.75in} \setlength{\parindent}{0.2in} \setlength{\parskip}{0.1in}

%\begin{document}

Shaky Greetings from L.A., \hfill December 1994\\

1994 started out with a bang! Then a crash and a boom. I love to "Shake, Rattle and Roll," but not with my house for a partner. Yes, the
Northridge earthquake really hit us hard, both literally and figuratively. It is almost 12 months later and we are only in the middle of
repairing all the damage done to the house. (And we were one of the very lucky ones-many people totally lost their home.)

Nasr did some travelling this year, but for once I'm glad I wasn't with him. He went to Russia and with all the stories he came back with, I'm
surprised we thought it was such a super power for so many years. He lived with no toilet paper, no soap, not enough food, and creatures
crawling in his bed. (The kids and I were merely living in a house covered in plastic and dripping with dry wall solution!!) Nasr looked in the
mirror a few months ago and realized he was starting to feel fat and look old. (Believe me, the grey hair has been around for a long time, but
he was trying to convince himself it made him look distinguished.) He went in search of his youth and found the UCLA gymnasium. He discovered a
new life of athletics, weight lifting and girl watching. He now feels as strong as Arnold Schwartzenegger. He has given up the idealism of
science and entered into the realism of Arnold. Wait for Nasr to be featured in his latest hit movie "Term Paper II."

Virginia's 1994 was like Mr. Toad's Wild Ride! In January the earth took us for several jarring and rolling rides. Thank God for our friends who
let us come to their house to take showers during that first horrible week, and thank you to all the wonderful people who called from all over
the world to make sure we were all right. To recover from the stress of the earthquake we were invited to Las Vegas to get away from it all. We
had the opportunity to do some horseback riding. The kids got on the pretty palomino, Goldie, and she walked beautifully around the area. Now,
Virginia was reminiscing how she used to ride horses as a child,and she wanted to get on, too. It had been 30 years since she was last on a
horse, but its like riding a bicycle, you never forget. Right? Wrong!!! Goldie is used to following another horse and when that horse took off
at full gallop, so did Goldie!! Screaming and crying did not stop the horse. Whoaing did not stop the horse. I watched the desert fly by me
because Goldie was hurdling small stone walls, medium size cacti and various sized boulders. She turned and I wasn't seated well and I did a
perfect 90 degree angle off her side, but I didn't fall. I realized the only one who was going to stop this creature was me. I pulled on the
reins with all my strength and I finally got the 1000 pounds of muscle under my control. Once I got back to base, (it took a while) I got off,
kissed the ground, and felt extremely proud that I was, at one point, in control. Then I had to go change my pants. In 1994 I also spent a lot
of time with my family. Poor Mom, she now realizes that none of her children have grown up and none of us can do dishes quietly. (But I'd never
admit that to my kids.)

Ami is only weeks away from 15 and driving, but she still has a year before she can solo. One more year before I turn grey. (No comments from
the peanut section.) Ami's motto is "Soccer is life, the rest is just detail." She is on the high school soccer team and on a tournament team
that is going to Hawaii this summer. (I'm going with her. It's a tough job, but someone has to do it.) Ami's whole focus is on soccer. Right now
when you ask what she going to do when she grows up, she'll tell you that she's going to be a professional soccer player. She has plans to get a
soccer scholarship to go college. That's ok with me as long as she gets the scholarship to Harvard or Yale. She will be the best educated
professional soccer player in the U.S. Then when she gets too old, like 28 or 29, her degree might still be good. Ami went to her first concert
this year and no longer do the girls swoon and faint just at the sight of the musicians, the musicians have to smash 3 keyboards on stage before
the audience is shouting for more. And now, she wonders why I worry. Haven't got a clue? Do you?

Adam at 12 is a changed child. Adam grew up. Adam transformed from a screaming, fighting, belligerent son of a Ghoniem to a well behaved young
gentleman. This is one of the very few good things that happened in 1994. He is studying on his own without any prodding, his grades are better,
and he takes good care of little sister. I don't know what happened, but whatever freak accident changed him to such a nice kid, I'm eternally
grateful. Adam is giving his dad a run for his money on the racquetball court. Nasr used to let Adam get a few points so he'd feel good, now its
Adam who is letting dad win. Nasr can still lift more weight, but if Adam keeps working out, Nasr may have to look for his youth in an old folks
home. Adam has a habit of falling on his fanny when we visit Wisconsin. We were there again this summer and Adam had a few scrapes. When Adam
was about 4 he fell off the wagon into the soft manure. This year he fell out of the second story of the barn onto the hard cement. Green Bay
hospitals can repair broken arms very well. That same morning, Adam was taking driving lessons with his dad. One of the two got too adventurous
and went onto the roads travelled by the Wisconsin State Police. Yes, he was stopped. Lucky for Adam the officer was in a hurry to help someone
else. Maybe his close scrape with the long arm of the law changed him or maybe he's just growing up.

Jasmine at 8 is still a horse fan. As far as the eye can see in her room, there are horses. Wherever we go we have to find a horse for Jazz to
ride. She's ridden in Egypt, Las Vegas and Colorado. Even though she never had any real lessons, she can handle a horse pretty well. She also
plays chess with Adam. At first she wanted to play with her big brother, but he didn't have the time to teach her. He gave her the instruction
manual and she came back 10 minutes later and she was able to play and she can compete pretty well with him. (Nasr said it took him 10 years to
learn to play.) Jasmine has turned into a great reader and continues to do very well with her art. Most kids just make their lunch and go to
school. Jazz has to decorate her lunch bag before she would consider putting anything into it. We always know what she's studying in school,
because her lunch bag has to coordinate with her studies, from reptiles to insects to mammals to yes, horses. If anyone is interested in
marketing lunch bag art, come to Jazz.

I was almost tempted not to write a letter because 1994 was such a bad year for us, but as I reminisce, there were some good times. And the good
times were because of our families and our very special, caring friends. Life has changed for us in Los Angeles. We're not taking anything for
granted and we're trying to improve the things we have. We try to appreciate each day as it comes. But no matter how bad it was, the great part
of 1994 is that our family came through the whole ordeal safe and sound.

I truly pray the coming year will bring good health and happiness to all. And no more rockin' and rollin'!!!!



%\end{document}
