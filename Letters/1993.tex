%\documentstyle[11pt]{article}
%\setlength{\topmargin}{-.75in} \setlength{\oddsidemargin}{-.3in} \setlength{\evensidemargin}{-.3in} \setlength{\textwidth}{7in}
%\setlength{\textheight}{9.75in} \setlength{\parindent}{0.2in} \setlength{\parskip}{0.1in}

%\begin{document}

Greetings and Felicitations of the Season! \hfill December 1993\\

Ok, I know felicitations is an archaic word but as time passes by I find that I'm getting more archaic also. (But under oath I will say that I
wrote that under utter duress.) This holiday season I tried to get my act together and get this letter out early. (My act is still in pieces so
you're getting this closer to New Years than Christmas.) This year we did a lot of things together as a family. And yes, we still are together.
There are times when family togetherness can break up the family, but we somehow managed to muddle through. Perhaps the unusual, scenic,
Mediterranean atmosphere helped us appreciate each other. But I think the real reason we all got along so well was because there were enough
adults around. We were able to pawn off the kids for a little while so they didn't drive us absolutely mad.

Nasr is at the age of forty-something now and we all love to remind him of his antiquity. He is still working at UCLA and he loves it (except
when he has to teach four hours a day!) Now those of us who are teachers of the younger persuasion (meaning we teach younger kids) teach six
hours a day, everyday, not every other day. Do you feel sorry for him? Me neither! But we do have to make allowances for those of advanced age.
Nasr also is the epitome of the absent-minded professor. The day after Thanksgiving I got my hair cut. It was very long, almost to the middle of
my back. I cut it to chin length. EVERYONE noticed it the minute they saw me. Nasr, sat down, told me about his computer problems, looked
straight at me for over an hour and didn't mention a thing. Finally Adam took pity on him (Adam saw the smoke coming from my ears) and told him
that I cut my hair. Nasr called down from upstairs, "Let me see this haircut. What does it look like?" Can we blame this on his antiquity too?
We spent the summer in Egypt and Nasr had a wonderful time with his family and friends. He spent hours in the water and swam like he did as a
child. I haven't seen him this relaxed in years. We enjoyed it so much that we want to go back every two years if monies allow!!


Virginia's age is really-something she doesn't wish to discuss. 1993 was a year of taking it easy. I wasn't taking classes at the university,
but I still managed to work full time and earn 12 credits taking Spanish classes and getting my Language Development Specialist Credential.
Sounds impressive? It's not. All it means is that I know how to teach English as a Second Language, and I get a little more money. Los Angeles
is back to a traditional school calendar. Last year was great, I had 8 weeks off in January and February and then about 3 months off in the
summer. This year I only get two weeks in December and I feel deprived. Egypt was great for me. It was the true vacation. I didn't clean, cook,
or care for the kids. I played, danced, swam and talked. Did I mention I didn't have to cook or clean or care for the kids? It was a real
culture shock for me when I came back to LA and I had to go back to being a mom and a responsible person. I guess I must have succeeded in being
responsible because this is how the three kids see me. Ami says I'm exciting and eccentric, Adam says I'm cool and Jasmine says I'm romantic.
It's funny, they said all those nice things and they didn't want anything. Maybe they are waiting to extract payment later. This could be a plot
to soften me up. Well, if at a weak moment, you see me open up my wallet and give them money or buy them something when they really don't
deserve it, just remember they pre-paid in compliments and it makes Mom feel good.


Ami, days away from 14, is the true scholar of the family. She graduated from junior high this 1 spring as valedictorian. In layman's terms,
that means she graduated with all A's and was number one in her class. I wish I could take the credit for all these brains, but I think her
absent minded professor dad gets a lot of credit for that. This year Ami is taking geometry. Now, when I was in high school, about a hundred
years ago, I took geometry. So one evening when Ami couldn't wake up her dad from a sound sleep (at 8:00 p.m.) to help her with her math, she
came to Mom for help. I thought I could help, but once she started stating all the postulates and theorems, I knew I was in BIG trouble.
(Remember, I'm the literary person in the family.) I took one look at those angles and I said "Those are angles that could maybe...
possibly...." Then my very intelligent daughter could read my mind and she said, "Oh, now I get it. This is the way it is done. " She went on to
explain in full detail exactly what needed to be done. I panicked as I waited for the second question. Thankfully, she could again could read my
mind after a few uhms and ahs. Next time though I"m waking Nasr up. Why should I panic when math comes so naturally to him? Ami is really into
soccer this year. She has been playing every other day all season. She played her last game in the pouring rain and miraculously didn't catch
pneumonia. She has been elected to play for the all-stars. She is very excited about this, but Mom was hoping she would get a rest from soccer
practice. Ami is also keeping up with her piano, but she is teaching herself. Ami liked Egypt, but she caught a miserable cough and it took four
months to get rid of it. She learned how to jet-ski in the Mediterranean and she can swim a great distance into the sea and back. One thing she
didn't appreciate was the fact that in Egypt the carcass of the cow is hung outside the door of the butcher shop or restaurant and pieces of
meat are removed as needed. I almost got her to eat lamb, in a restaurant, in the desert, outside of El-Alamain, until she saw the meat being
carried in a pink, plastic bag on top of a car through the Sahara. After that, she wouldn't even walk past the meat. She went out the back door
just to avoid the meat hanging in the doorway. But she had fun in Egypt!!!

Adam is eleven and dating. Yes, I said dating. However, this was done without Mom's approval or KNOWLEDGE. The story he told Mom was that he was
going out with his friend. I found out several hours later, that he was at the movies double dating with his friend. If this was a Saturday
matinee, I might have been slightly miffed, but this was a Friday night and he walked down a major street in LOS ANGELES with only his friends.
He survived to talk about it, but I warned him if he even thought about dating before he was 21 I'd do something drastic. (I know he's already
thought about it, so if you have any good suggestions of something drastic that I can do, please let me know. ) Adam is in the 6th grade and is
in junior high. It is hard for me to imagine that Adam is in junior high, but he is very tall and all too soon, he will be towering over me.
Adam is becoming much more helpful around the house when the mood strikes. It doesn't strike very often, but when it does he manages to fix
things that Nasr has been avoiding. He fixed the outdoor lamp by his basketball hoop that Nasr ignored for a year. Do you think he had an
ulterior motive? Adam is still at the dreaming stage. Each week he is going to be something different when he grows up. He wanted to be a film
producer and movie-maker. The fact that they are filming a movie at the end of our street may play some role in the interest he has in this. He
talked for a long time with the producer of this movie and he wanted to make great movies, too. As of today, he wants to be an astronaut and an
engineer like his dad. He wants to take over his dad's company once Dad has made it big. Adam's got the business mentality. He's a little
concerned about being an astronaut because they have to learn to be in isolation for long periods of time when they are travelling through
space.

He's not so sure he can handle that because when he was in isolation for dating, he wasn't happy at all. Adam is also becoming very musical.
He's trying to take up the guitar and piano by teaching himself. He is a little more successful so far with the piano. He keeps hinting that he
wants to take guitar and piano lessons after Christmas, but I'm not so sure he'll stay with them. I like the idea that he is taking the
initiative and doing it on his own. I even offered to help, but it seems that the times when he wants help and the time that I'm available to
help him don't coincide. Adam also had a great time in Egypt. Adam is under the belief that whatever Nasr can do, he can do better, especially
when it comes to swimming in the Mediterranean. Nasr would swim out into the very deep water and Adam was his shadow. It scared me half to
death, but Adam didn't stop to think that he was in water 50 feet deep or more. Maybe he's used to being in deep HOT water with Mom.

Jasmine is seven going on 37. She is very grown up for her age and tries to use the most sophisticated language. However, it is her own
language. Like her mom, she finds great refuge in the "frig-derator". It is amazing how all the cookies and chips in the house seem to
"sip-appear." She loves to have meals of "piss-getti and hambugers". I have a hot financial tip for all of you who follow the stock market. Make
sure you invest in any and every toy company that makes toy horses. If the amount of money we've spent on these creatures is any indication,
these companies are doing fantastic and making gobs of money. You guessed it, Jazz likes horses. You also might be wise to invest in paper and
crayon companies because if Jazz is not playing with the horses, she is drawing them. We received, as a gift, a picture of horses done in silk
that was painted by the daughter of the last emperor of China. Now, Jasmine, having such a recently famous name as Princess Jasmine, has decided
that her horse drawings are going to be as famous as the emperor's daughter's drawing. Let's hope! Jasmine also rode her own Arabian horse
around the Pyramids in the Sahara desert. But I don't think that it compared to the two painted and flocked ponies that came to the house for
her birthday. She rode them with the greatest of pride and tried unsuccessfully to convince Mom to buy her one. I know our house is zoned for
horses, but I know who would have to take care of them, and Grandpa just won't give up his lifestyle in the cold, Wisconsin winters to take care
of these creatures. Or will you, Daddy?

This year has been an interesting and fulfilling year. Last week on my way to yet another test, (yes, I've been tested a lot this year in more
ways than one) I watched the haze of a cloudy sunrise over the mountains of the Angeles National Forest. I was listening to Christmas carols in
the new car (it's a year and half old, but it's new) and the haze almost looked like a thin blanket of snow. Maybe I wanted to see snow, not the
California haze, yet it gave me a real Christmas feeling. Sometimes in "sunny" California we need to feel the old feelings of Christmas and
home. No matter how old we are we will always see Christmas through the eyes of our childhood and mine was snowcovered, fun-filled and loving.
I'm hoping to inspire those good feelings in my children, but our children will see Christmas as a wildfire of flooding, landsliding happiness
and earthshaking love. I hope all of you will have your own special Christmas and your own wonderful memories that make each Christmas unique
and unforgettable. Merry Christmas and Happy New Year. Here's to 1994! May it bring everything you desire and deserve?

Lots of love,//

Virginia, Nasr, Amira, Adam, Jasmine, Butterscotch (the dog), Jack (the bird), Sharky (the fish) and too many nameless, colorful rabbits.


%\end{document}
