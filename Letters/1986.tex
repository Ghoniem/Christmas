%\documentstyle[11pt]{article}
%\setlength{\topmargin}{-.75in} \setlength{\oddsidemargin}{-.3in} \setlength{\evensidemargin}{-.3in} \setlength{\textwidth}{7in}
%\setlength{\textheight}{9.75in} \setlength{\parindent}{0.2in} \setlength{\parskip}{0.1in}

%\begin{document}

Greetings from the Ghoniems\hfill December 1986\\

MERRY CHRISTMAS TO ALL! "  ... I apologize in advance to all the people who hate getting this kind of Christmas letter.  Every Christmas is a
busy time of year, but as of yet, I haven't adjusted to the changes in our family life.  Give me another 18 years and I should have it down pat.
So, as I was saying, any of you who can't stand to read this, stop right now and give us a call.  I promise to give you a very personal
rendition of what has been happening to us over the past year.  The number is (818) 891-1616.

Now, for all you die-hards who are willing to put up with this kind of letter, or for those of you who actually enjoy them, I will continue. The
biggest news about the Ghoniems is that we have expanded.  I suffered all through 1986 (not too silently either, just ask Nasr) in order that we
could have our third and LAST child. (Yes, you may quote me on that).  Jasmine Suzanne Ghoniem was born on August 29, 1986 at 11:46p.m. She came
into the world at night and that is the time of day she loves best.   She weighed 8 pounds 7 ounces at birth, and now has topped the scales at
over 13 pounds.  She is growing fast and takes after her mother, she doesn't suffer in silence. I think Jasmine may be our only blue eyed baby,
finally one of the kids took something from Mom.

Adam is four years old and is not adjusting very well to having a baby sister.  He spends the majority of the day trying to twist his hand
around her head as if he were trying to juice it like an orange. To say the least, Adam and I are having quite a few arguments over the proper
way to touch the baby, and I'm losing all of them.  Adam started pre-school this year and goes three days a week.  He enjoys it and it gives him
some socialization with other children, and besides all those good things, it gives Jasmine's head a rest.

Amira is six, about to be seven, but going on thirty.  She is doing very well in first grade.  She has been accepted into the gifted program.
Now all these talents come from someone, and depending on whether you ask Nasr or me, will determine whether you hear it comes from my side of
the family or his.  But since I'm writing this letter, it comes from the Schink side of the family.  Amira is an excellent baby-sitter.  Once
she finishes her homework, I simply drop Jasmine into her lap and then proceed to do everything I should have done during the day in
approximately one hour.

Virginia, still young enough to admit she's only a mere 32.  Look, what do I have to worry about, I'm the fifth out of six.  It's my siblings
that cringe when I give my age because everyone knows they're lots older than me.  (How many of you think I'm going to get four very nasty
letters from my siblings? At least that's one way to get them to write to me.)  To say the least, I've been very busy with the three kids.  I
spend the majority of my day chauffeuring kids to and from school, settling fights, helping with homework, nursing and changing babies.  My
writing is sort of by the wayside until Jasmine is a little older and I'm more organized.  We are in the process of remodelling the house, and I
must admit, IT IS DRIVING ME CRAZY!  We made the mistake of having a child at the same time as we remodelled.  Learn from someone who knows,
either have a child or remodel, don't do both!

Nasr, who will really be 39 in January , and I'm sure he'll stay that age forever, has also had a busy year.    He's had several opportunities
to move, but found UCLA really is the best for him.  He became a full professor in July and now theoretically, he could sit back on his laurels
and relax, but that's not in Nasr's nature.  He has his hands in more pies than ever now and he almost has me convinced that he likes what he's
doing.  But then again, our trip to Lake Tahoe showed him how good the life of leisure really is.  He faltered a little in the summer, but now
he's back full swing and working long, hard hours.  Nasr, who loves large families, is definitely convinced that our family is large enough. The
noise level in the house is deafening and most days, he would like to escape to a quiet place (for that matter so would I), but he's a very good
daddy who spends lots of time with all three kids, and no, he hasn't forgotten how to change a diaper.

Our year has been filled with joy, happiness, and new beginnings. It is our fondest wish that all of our friends and relatives have a very Merry
Christmas and most of all, a wonderful, Happy New Year!


%\end{document}
