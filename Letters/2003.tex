

Happy Holidays!                                                  \hfill December 2003
\vspace{.5in}
Yesterday was Thanksgiving and today our bellies are stuffed and our souls are overflowing with thankfulness for our wonderful friends and
family.  Soooo, I cranked up the Christmas music (Nasr said not to play it too loudly-I ignored him) and I decided this was the perfect time to
expound on why we are so thankful.  (It may make you thankful that you don't have to deal with our trials and tribulations.)

Amira (24-or oh so close) (Pretty soon I'm going to stop posting their ages because there are too many smart people out there who can do the
math about our ages.)  This has been a very exciting year for Amira.  Nasr and I are particularly excited because as of May she is fully and
gainfully employed and quite an independent young lady.  (Ok, so they are never completely financially independent, but we are paying a lot less
of her everyday bills.)  In June, she earned her Masters Degree in Electrical Engineering from Stanford. The ceremony was spectacular.  I don't
know who was more proud, Amira or Nasr. While she was a student, Amira met a fine, handsome, young man, Matthew Koerner.  He is also an
electrical engineer and works in the same company as Amira.  Matt recognized what we already knew, and Amira realized that she found a good man.
On October 3, 2003, Matt proposed and surprise surprise, Amira accepted.  Once they told us, we were jumping for joy and then suddenly it hit
us.  WE HAVE TO PLAN A WEDDING!   The first thought was to have a June wedding.  Amira wanted to relax and enjoy being engaged-and I heard the
clock ticking  (actually it was banging) and I could feel all the venues for a ceremony and reception hurriedly slamming the doors in our faces.
We found some spectacular places and were they spectacularly expensive!  Finally, the young couple found the most perfect place to get married.
They are getting married in a castle garden in Beverly Hills and partying in a penthouse overlooking the Pacific Ocean and the sparkling lights
of Los Angeles.  And if you think this is happening in June, think again.  The wedding is set for September 19, 2004.  Next year, I'll fill you
in on the trials and tribulations of a wedding. (We are already talking to our friends who will make sure we don't make the same mistakes they
did, but I'm sure, we'll make new ones.)

Adam (21) has decided he wants to compete in the Mr. Universe contest. (Only kidding!)  But, he has gotten on the exercise bandwagon. This
summer, he encouraged Nasr and I to get on our bikes at 5:00 a.m. and ride 4 miles one way to the gym, then we worked out for an hour and then
proceeded to ride our bikes back home for another 4 miles.  For Adam it was easy, for Nasr and I it was exhausting, but boy did we feel a sense
of accomplishment as we huffed and puffed our way uphill.  In the process, Adam is building muscles and sculpting his body.  He has been
attending college and working in the engineering department at UCLA.   Adam is slowly taking on the cleaning traits of his father. Almost every
morning one can hear the loud, disgusted, frustrated, voice of Adam as he tells Jasmine that she needs to clean up the bathroom.  He admits that
he has 1/10 of his father's neatness bug, but he figures that when he gets be his dad's age, he will be just like him. (I'll make sure that I
warn his future wife.) Money continues to burn a hole in Adam's pocket. I think when Adam was young and Nasr explained the difference between
saving and spending, Adam mixed them up. Even though he can't save a dime, he has an overabundance of patriotism and wants to help his country.
His patriotism mixed with all the exercise helped him decide to join the U.S. Marine Corps Reserves.  He will be leaving for Boot Camp on
December 15th and Nasr and I will be in San Diego  in March to see his ceremonious graduation where he will be marching to a different drummer.
Our cats bring many interesting and unusual gifts to us, but last night Luna carried a small ornament-wrapped like a gift-to Adam's room.  Does
she know Adam won't be home for Christmas?  I told Adam this "gift" would bring him good luck. But, there's no way I'll unwrap it, because Luna
is a very mischievous cat.

Jasmine (17) is in her senior year in high school.  (Need I say anything more?)  Can you believe that little Jasmine is going to be finishing
high school this year? She can't wait to turn 18 and move out. (You know what I'm not saying, don't you.) She is taking a very light load
academically, but she is taking on a very heavy load in the work world.  She continues to be a sandwich artist at Subway.  She is learning that
earning minimum wage is not what she wants to do and it is encouraging her to want to do well in college. (However, she hasn't connected
studying in high school as a precursor to doing well in college.)  As all children in their senior year, Jazz is expected to make some major
decisions about her life-like what will she study?  What will she want to do when she has finished college?  She made it to the "D's" in the
alphabet.  Right now it is a choice between a dentist and a designer.  Clothes with designer labels are the only clothes that Jasmine will buy.
Now that she is working, she does have money to spend.  We try to tell her that she wants to save money to buy a car.  (The car she is driving
is 10 years old.) However, like her brother, she does not see the value of saving.  She does however, see the value of a name.  Like Jasmine, I
have to admit the designer  names of Von Dutch and Bebe make me shake.  But, Jazz is shaking with excitement, while I'm shaking trying to not
strangle my daughter for paying \$300.00 for a jogging suit with a suggestive word written across the derriere.  Jasmine has a specific, subtle,
sense of beauty.  I will agree that everyone feels certain parts of his or her body can accentuate how good you look.  However, for Jasmine, she
is totally paralyzed and cannot go anywhere, or even take a shower, if her eyebrows are not perfectly shaped or groomed.  Jazz says, "Eyebrows
are everything and if they are shaggy or grown out, you look disgusting, just ask anyone."  So I'm asking.  Are eyebrows everything?  Where was
I when the fashion people decided that eyebrows were the most important feature on the human body?  Maybe it all goes back to the beginning,
Jasmine is a senior in high school.  Need I say more?

Virginia (Glad she is older than 17-not admitting to much more) is planning a wedding and she will not have anything else on her mind for the
next year.  (ARRGGGHHH)   I am calm about the whole thing, but don't ask Amira or Nasr because they have very different views on how calm is
expressed.  Ok, I am working, so I guess I will have a few more things on my mind. (At least during working hours.) I have changed schools and
now I am the principal at a school only 6 minutes from my house. (I have to be very careful that I change that part of the letter when I mail it
to my friends in Los Angeles because most people commute much further.)  Not only am I closer, I got a raise-hey what a deal! (Thanks Boss!)
The biggest perk about moving to the new school is that I got the chance to meet Kirk Douglas.  I shook his hand, I introduced him to the crowd
and I have to tell you, he is a very wonderful, genuine, human being.  And, (sorry Nasr) even at his age, he is still very handsome.  I grew up
watching Kirk Douglas on the big screen and I can attest that he is one man that didn't need any special make-up.  There was one point a few
weeks ago when Nasr was getting a little worried when I was wearing a picture of Kirk on my forehead.  But he has relaxed a little, now that I
moved Kirk to the refrigerator.  This summer Nasr and I escaped-oops I mean, traveled to Colorado to see my brother, Steve and his wife, Gayle.
We did the usual strenuous exercise of lifting our hands from the plate to our mouths while laughing and having a good time, but Gayle and I
decided to be even more adventurous and go for a horseback ride in the Rocky Mountains.  (It was supposed to be a flat ride along the river.)
However, it turned out to be a two-hour ride straight UP and straight DOWN the mountain.  Now I know why the cowboys hug their horses.  They
aren't really hugging them, they are hanging on for dear life because their horse is going up a mountain goat trail.  Yes, I rode UP this
unbelievably steep mountain goat trail.  Then, when the horse was taking a nosedive down the mountain, I laid back on its back and prayed to God
that I would make it out of there. Hey, if I can ride those mountains at my age, and not fall off the horse, I can do anything.  I can even plan
a wedding.  Adam got Nasr and I started on the morning exercise.  Nasr and I still get up at 5:00 and work out for an hour.  (We work, so we
have to use horsepower instead of leg power-i.e. we drive instead of ride bike.)  But, we both enjoy it.  Adam, however, lost interest in the
early morning exercise.  He has changed his exercise to the evenings.  I think he figures he'll get enough exercise early in the morning in the
Marines.

Nasr, (I promised not to make a comment regarding his advanced age this year. Oops, was that a comment?) is taking on new responsibilities all
the time.  He is father of the bride, Vice Chairman of his department, traveling, running a large lab, supervising 20 students, teaching two
classes, traveling, chauffeuring Adam to wherever he wants to go, and trying to put together a business. (He and Adam are making BIG plans for
the future.)  Do I see my husband?  Occasionally.  Actually, Adam and Nasr see a great deal of each other.  Adam does not drive so this has
given Adam and Nasr a great opportunity to bond.  Nasr may complain about how busy he is, but he thoroughly enjoys the long talks and talk of
dreams that he and Adam share on the long ride to UCLA and Pierce College.  Did you note that I wrote Father of the Bride first?  Nasr and I see
our roles differently.  I see the wedding taking up all my energy, Nasr sees it as an exercise in check writing. He thinks he will have to go to
the gym twice a day to lift weights so he can handle the burden. (I made the mistake of taking him to the Ritz to help me price the meal.  He
has almost recovered.)   Yes, Nasr traveled again this summer.  As usual he had a great time in Egypt with his family.  He also went to Greece
and spent a day on a Greek Island.  (Oh yes, this was one of his conferences from which he comes home very exhausted from all the work.) A group
of scientists from America, Greece, Egypt, China, Russia and Spain decided to play hooky from the conference.  Luckily they had a Greek with
them who was their tour guide.  They toured the ancient birthplace of Alexander the Great in Macadonia, swam in the warm, turquoise-blue water
of the Mediterranean, and then had a traditional fish lunch, interspersed with laughter, intellectual thought and relaxing drinks.  They
continued their tour of the Helkidikki peninsula where they saw ancient ruins and picturesque mountains, swam in a secluded water inlet and felt
the warm Mediterranean breeze waft over them.  As the day grew to an end and they watched the lights of the ships, and the nestled homes
twinkling beneath them, the group decided that they had all experienced the perfect day.   Never, EVER, feel sorry for Nasr when he says he has
to travel.

As the year comes to a close, and we all remember the good, the bad, and the not so pretty, it is our hope that next year all our friends get
the chance to experience their perfect day over and over again.  I hope your holidays are special and you have a very Happy and Healthy New
Year!

