%\documentstyle[11pt]{article}
%\setlength{\topmargin}{-.75in} \setlength{\oddsidemargin}{-.3in} \setlength{\evensidemargin}{-.3in} \setlength{\textwidth}{7in}
%\setlength{\textheight}{9.75in} \setlength{\parindent}{0.2in} \setlength{\parskip}{0.1in}

%\begin{document}

SEASON'S GREETINGS! \hfill December 1991\\

Can you believe that I wrote "Seasoned" greetings? That Freudian slip is very appropriate. The Ghoniem life is seasoned with fanciful and exotic
spices and interspersed with coffee dregs. But, it's been a good year for us two "seasoned" veterans of child rearing. Each year the seasoning
gives us more wisdom and we are able to philosophize about life. We've learned why children are so fascinated with the Merry-Go-Round. It
teaches them when they are our age, uhm-over 21, that life really is a merry-go-round. However, I better teach my children how to get off it.
Right now I'm riding the confounded thing and it doesn't stop, and I can't get off. Anyone who has any good suggestions on how to get off,
painlessly, please write.

NASR, over 21 a couple of times plus, has had a great year. He took a six month sabbatical and spent it in Santa Barbara. This would have been
fine with me, but I found out that the engineering building is only a quick hop to the beach. He spent his lunch hours studying on the sand.
However, the only structural analysis he was making was on the young co-eds in their bikinis. Don't you feel sorry for a guy who has to do such
gruelling tasks? Also, Nasr's parents spent a wonderful and helpful 6 months with us. Nasr got the chance, for the first time, to do some
carpentry work with his father on projects for the house. When it was all over, they were like identical twins beaming with pride. Nasr is
busier than ever this year teaching and consulting. His merry-go-round is keeping him so busy that we sometimes wave as we keep circling.

VIRGINIA, at least 21, is busy working full time at Gault Street as the afternoon kindergarten teacher. I'm finished with my clear credential!
YEAH! Last week was the greatest thrill of my "seasoned" life. A waiter, (half my age) wanted to make sure I was 21. I was so pleased that he
got a kiss. I think he'll be more careful who he cards from now on. My writing has only been academic papers, but I'm hoping next semester I'll
have a chance to be creative again. I'll fit it around the several Spanish courses I'll be taking so I can communicate with the children in my
classroom. I've also set a goal for second semester. My goal is to make dinner two nights a week. If you ask Nasr, (be careful, he may grumble
and growl) he'll tell you that I only cooked two meals all semester. Whoever invented already prepared frozen foods will be in my will-well
she's at least in my prayers because she gets enough of my paycheck as it is.

AMIRA will be 12 on January 6. She still prefers to be called Ami, but Mom at least has to use her official name once a year. She is in the 7th
grade and taking honors classes. It's not that I'm proud of my daughter, but she got straight A's in her honors classes. (Of course, we all know
where she got her intelligence from. Be careful how you answer that!) We say thank God she has the capabilities and pray she'll do something
profitable with them. (When she was 5 she promised to buy Nasr and I a Mercedes when she got out of college. Hopefully, she'll use her brains to
make lots of money. I'm waiting for my Mercedes. Remember, I'm a teacher; I can't buy it on my salary.) Ami was in soccer again this year and
her team did quite well. Ami is also good in athletics and won the sportsmanship trophy. Ami is successful because she has a method for
everything. Ami's bed is always made and her method for keeping her bed made is that she never sleeps under the covers. She grabs an extra
blanket, suffers by being cold, but mind you, her bed is ALWAYS made.

ADAM is 9 and in the fourth grade. He is doing great in school. He is on the student council, involved in lots of activities and is even in 5th
grade math. Adam is one of the best players on his soccer team and now they have made it to the play offs. Adam also won the sportsmanship
trophy. (Mom's merry-go-round will slow down a little now that she's not chauffeuring to soccer 4 nights a week. YEAH! Oops, that slipped out.)
Adam is a very environmentally concerned child. He knows more about the environment and what horrible things we adults are doing to it. He has
big plans to clean up the earth and make it a better place to live. (I hope he's more successful at that than he is with cleaning up his room.
That is an environmental hazard.) Once Adam gets engrossed in a project or in many cases, gets mesmerized by the television, he has a hard time
focusing on what he should be doing. I'll know because many a day I'll find the legos in the refrigerator and the food on the counter, or find
the t.v. remote control in the cupboard next to his chips. But even though he sometimes has a hard time focussing on the issue at hand, other
times he hits it right on the head, or in this case, right on his Dad's head. He looked his Dad straight in the eye and said, "Daddy, is the
shiny spot on your head the light from the light bulbs inside your brain when you get an idea." (Some people might call it a bald spot.)

JASMINE is 5 and going on 35! She's in kindergarten and according to Jazz, life in kindergarten is great! She's now a game leader and she's
losing a tooth. Those are the two most important aspects of kindergarten. Last Easter we got chicks and bunnies for the children. Well, bunnies
turn into RABBITS, and chicks turn into CHICKENS. I never could acquire any affection for chickens, but Jazz adopted these two birds and loves
them dearly. She hugs grown chickens like I would hug a puppy. She clucks to them and yes, they do answer her back. They seem content to be
squashed onto the chaise lounge so Jasmine can have their full attention. Jazz is into the question stage of her life. One question I got was,
"Who is Mississippi?" "It's a big river," I answered. She frowned, looked very puzzled and then said, "Oh. I thought there was a teacher named
Miss Issippi."

One Saturday, not too long ago, the house desperately needed to be cleaned. (Nasr was coming home from one of his trips and I didn't want him to
see how we really live when he's away.) The only way the cleaning was going to get done was if the children helped to clean it. I put on my 50's
and 60's music very loud. (I think the loud part pleased the children the most.) And miracles of miracles, all three children started cleaning
and they cleaned WITHOUT fighting. Adam was enjoying the music so much that he grabbed a broom and started dancing with it. Ami took the vacuum
cleaner hose and used it as a microphone. She put on a great show. Jazz danced and was too busy having fun to mess up what was already cleaned
up. Maybe the conservatives from the 50's were right when they said there was some sort of subliminal messages in the music. If it made my
children clean without fighting, hey I'm all for it. This letter is long, but a lot happens in a year. I wish I could sit with everyone of you
and hear your stories. Friends and family are precious gifts and I wish I could spend more time with each one of you. But remember, if you can
read this, you can also write us and tell us your news or call on the phone. Jasmine has a very clear understanding of how useful a phone is.
She knows it allows you to communicate with important people. One of her questions was, "What's God's telephone number?" I don't have God's
number, but I do try to stay in touch. I hope all of you are blessed with good health and happiness and by the way, if anyone does know God's
number, please let me know what it is.

HAPPY NEW YEAR!!!!





%\end{document}
