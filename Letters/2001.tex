%\documentstyle[11pt]{article}
%\setlength{\topmargin}{-.75in} \setlength{\oddsidemargin}{-.3in} \setlength{\evensidemargin}{-.3in} \setlength{\textwidth}{7in}
%\setlength{\textheight}{9.75in} \setlength{\parindent}{0.2in} \setlength{\parskip}{0.1in}

%\begin{document}


Greetings of the Season Once Again!\hfill December 2001

2001 has been a year of personal and national trials.  We'd like to believe that these trials make us stronger, not just a little crazier.

Nasr had a traveled-filled year up until September.  He traveled throughout the United States, as usual, but this year he got the chance to
spend the summer with his family.  The definition of relaxation for Nasr is Alexandria, Egypt.  It means strolling along the Mediterranean Sea,
sounds of laughter from his children, catching up about family and friends with his father and mother, and napping in the afternoon with the
breeze from the Mediterranean wafting across his body. Nasr's schedule is also a relatively relaxing one. It allows him to teach his classes,
have office hours and work with this research group.  But there are still a lot of hours in the day from 6:00 a.m. to 11:p.m. for him to sit at
his laptop computer and write proposals and books and still come up with lots of ideas for more research. He has also taken on the
responsibility of helping Jasmine.   He has decided he could be a professional high school student. (He's only helped three of them through!) He
reads the books and then re-teaches the math and chemistry to Jazz.  Too often, I find him curling up with her books every night before he
sleeps because he has to re-learn the way they teach it here.  It's hard staying one step ahead of your teenage daughter and to try to stay
several steps ahead of the research pack.

Virginia had an eventful year, one of personal health issues and fears.  Thank goodness the health issues were solved by surgery. The nurses at
the hospital informed me that one night I was the super-hero, "Hot-flash," who destroys the enemy with rays of heat that shoot from her palms.
I tried to demonstrate my powers, but I was too doped up for the rays to work, and unfortunately they gave me new medication the next day.
After surgery you need to recuperate. The best place for that is Egypt. We were fed very well, given transportation and had no responsibilities.
The kids found friends and I saw them for a half an hour a day between 2:00p.m. and 2:30p.m.(Dinner time--their breakfast).  I read books, swam
in the warm, soothing Mediterranean, laughed with friends and did nothing.  I had thoughts of doing something, but I waited 3 minutes and the
thoughts disappeared.  My niece's wedding in Maine was a great family reunion. Portland barely survived, but we had lots of fun.  The other
guests at the wedding wanted to know if we would "work" other weddings because we certainly know how to get everyone partying.  The bubble juice
may never wash out of my hair, but I certainly had fun flooding my siblings with bubbles. (For some reason, other nieces and nephews are
concerned about inviting all the aunts and uncles to their weddings.)  In between all my adventures, I do manage to be a principal at one of the
high performing schools in California.  It keeps me very busy and Nasr's favorite thing to say is.  "You are too busy for�.." (Just fill in the
blank.)  I always argue that I'm no busier than he is, but after 25 years of marriage, I realize this is an argument that I will never win. But
I did get to retaliate by stuffing cake in his face at our 25th Anniversary party. 

Amira, 22, graduated from UCLA as an electrical engineer.
Before she left for Stanford, she spent the summer doing her internship at Aerospace Corporation and visiting Egypt and London. In Egypt she was
sunbathing, night clubbing and "hanging" with her friends until the wee hours of the morning.  As the plane took off for London, she was making
plans for a visit to Egypt in December. (Not this year! Mama says.) In London, Ami took on the responsibility of planning our itinerary.  We
toured the hallowed halls of Buckingham Palace, Windsor Castle, Hampton Court, and the Tower of London.  If Ami wondered what an engineering
degree would get her, I think these castles opened up a few possibilities for her.  Once we returned from our trip, there was a lot of last
minute shopping to get both Ami and Adam ready to move into apartments. I made an executive decision.  Ami got the new things because Adam was
moving in with three other guys.  Moving Ami up to school was tough.  She moved a lot of her things on Thursday, but Nasr and I brought the
remainder on Saturday. We found her surrounded by boxes and computer peripheries unsure if she would have room for it all.  Four of us unpacked
for a day and shopped for another. She is doing quite well.  I am sort of surviving.  Ami has incorporated her Egyptian culture into the
Stanford culture.  Here belly-dancing speech, complete with finger cymbals and veils, got rave reviews. The Phone Company loves Ami.  Between
her calls home and mine to her, the Phone Company will be able to expand into every third world country, by the end of 2001!

Adam, 19, is in his second year at Santa Barbara.  It was touch and go for a while at the end of his first year, but he has pulled himself up by
the bootstraps and is doing very well. I think sleeping in a van with two flat tires while he was on spring break made him realize being a bum
isn't glamorous. He figured getting an engineering degree is a lot easier than not having a roof over your head. While visiting friends, Adam
and two other guests got locked out on a balcony overlooking a large hill.  The owners of the house were asleep downstairs and being a
thoughtful guest, Adam did not want to wake them up.  He decided to take action by throwing a rock. Not through the door, but over the balcony.
He measured the time it took to reach the ground. He then put together an equation based on the rock time, his height, including the length of
his very long arms. (Adam was the tallest--or maybe the others didn't want to put their life into Adam's equation.)  Based on this formula, he
decided he could drop off the balcony and not break his legs.  It worked.  His legs are intact and his friends are no longer stuck on the
balcony. Adam will never say "Why am I studying physics?  I will never use it in real life."  You may want to tell this true story to get your
kids to study physics.

Jasmine at 15 is into driving.  She had her first behind the wheel driving experience in driving rain. She wants to take her permit test before
she has finished with her driver education classes.  Why is Mom so mean? No one else finishes the classes before they take the test! Once Nasr
brings Jasmine home from school, she runs for the phone and talks to her friends while munching on a snack.  As soon as Nasr figures she has had
enough conversation with her friends, he calls her on his very first cell phone.   Nasr finally entered the 21st century. With call waiting,
Jazz won't miss any phone call. She picks up the line and Nasr calmly tells her she needs to start studying.  It works quite well and it saves
Nasr's throat and lungs from unnecessary torment. Jazz is back to doing her artwork.  She makes beautiful faces of characters and we have her
acrylic water scene hung on the wall.  This year Jasmine is doing so well in school.  I am not complaining, but the child we had last year is
not the child we have this year.  I think aliens took her last year and gave us someone that we did not know.  This summer, they brought our
wonderful Jazzy back to us.  We are so happy and I believe she is, too.

The emergency number 911 has taken on a new meaning for everyone in the United States.  Everyone suffered.  Muslim Americans felt the painful
brunt of stereotyping and we all felt the frustration of being unable to help those you love. Let's pray for better years to come.  Happy
Holidays! God Bless America!  And to quote someone who dealt with adversity and challenge, "God Bless Us All, Everyone!"


%\end{document}
