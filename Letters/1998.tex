%\documentstyle[11pt]{article}
%\setlength{\topmargin}{-.75in} \setlength{\oddsidemargin}{-.3in} \setlength{\evensidemargin}{-.3in} \setlength{\textwidth}{7in}
%\setlength{\textheight}{9.75in} \setlength{\parindent}{0.2in} \setlength{\parskip}{0.1in}

%\begin{document}

Greetings from the Ghoniems\hfill December 1998

This has been a year of celebrations and sadness.

Nasr celebrated his 50th birthday with a great deal of sadness, but it didn't stop him from enjoying his sabbatical. Believe it or not, I think
he worked more during his sabbatical than he is working this year. (The guy is kind of backward.) At the end of his sabbatical, we took a
"Griswold" (Ghoniem) family vacation. Yes, we drove across the US, but we did it Ghoniem style. We invited Nasr's brother, Gamal and his family
to take their vacation with us. We had a great time, but Nasr wanted to see the entire Western United States in less than two weeks. (We almost
accomplished it!) My sister-in-law, Magda took the three teenage boys in their car and gave me the three teenage girls. (I think she earned a
place in heaven for that.) You must picture two very large, sports-utility-vehicles crossing the wide, open expanse of the desert. At the
wheels, were two eager men clutching their miniature Citizen Band (CB) radios and reporting any evidence of police in the area. Sitting in the
passenger seats were two bleary eyed women reading maps, watching for road signs, and censoring the use of the men's CB radio. DO NOT ask Nasr
to tell his 2-2-2 joke. The girls read magazines and the boys slept while we traveled from Los Angeles, to Laughlin, Nevada, to the Grand
Canyon, to Sedona, Arizona, to Las Vegas, to Bakersfield, Ca.and onto Yosemite National Park. If Nasr hadn't gotten ill, we would have continued
on for another three days and visited 4 more cities along the California coast. (Lesson learned: Do not let Nasr plan any future trips because
he wants to do it all in 30 seconds or less.) Nasr had so much fun traveling this summer (or maybe he had too much "togetherness" with his wife
and kids), that he is traveling almost every month this school year. Since September, he has been to Spain and Chile and before I write to you
next year, he will travel to Mexico, Japan, Singapore, Hong Kong, Portugal, Denmark and maybe Egypt. Don't you feel sorry for him? At least feel
sad for me because I can't go.

Virginia is 21 and holding. I finished my Master's Degree the end of July, but I graduated in May. (Figure that one out.) Anyone who has seen my
graduation picture comments that my uncontrollable smile says it all. I'M DONE!!! I'M DONE!!!! I'M DONE!!!!!!!!!    Yes, I've been celebrating
the end of a tough year. To tell you the truth, I have not been looking very diligently for a job in which I can use my masters, because once I
get a vice-principal position, I have to go back to school for another 24 credits. AAAAGGGHHH. Educators are constantly being educated. I guess
they believe the statement that one can never know enough. Sunday morning is the time to clean the Ghoniem house and being the educator that I
am, I cannot let any educable moment pass. So, while the family is cleaning, we have a music history lesson. We listen to the music from the
70's, 60's and 50's. Of course, I have to instruct them in the proper way to dance while listening to these tunes. The children roll their eyes
and continue dusting, but they are definitely picking up speed with their cleaning duties. By the time we are listening to the 50's CD, I am
using the central vacuum hose as my electric guitar and am singing "Reelin and Rockin" by Chuck Berry. It is at this point that the kids are
shaking their heads and wondering about the sanity of their mother. They are also extremely thankful that the house can be cleaned within the
time span of three CD's.

Ami will be celebrating her 19th birthday at the beginning of January, and is doing very well in her second year at UCLA. We had her all set up
to live in an \$800 a month apartment for September. She moved back home in June and organized her room. This was the first time since the
earthquake that she put any real decorations on her wall. When it was all finished, she really liked the way it looked. (I do too, but don't
tell her because it might scare her.) In early July she came up to me and asked "Why do I want to share a bedroom half the size of mine with two
other people?" I very coyly shrugged my shoulders and whispered, "I don't know?" She now lives at home and commutes by bus. Mom is celebrating
having her close to home and Dad is celebrating not having to pay the extra \$800 a month. Of course, we did entice her with a car as another
reason to stay at home. But, during the summer she drove to her job at UCLA and decided she'd much prefer to take the bus rather than fight the
daily LA traffic. Ami does sadly wonder though, if she will ever see the car that we have promised her.   Soccer is still a big part of Ami's
life. She is continuing to participate in and win soccer tournaments. One of her requirements for college is a health course and she is learning
how important it is to eat right. She is putting everything she has learned to practice and is looking great. I'm glad my kids do as I say, not
as I do.

Adam turned 16 this summer and is celebrating his ability to drive, but sadly wonders why he has to drive the "Poo-Mobile" (AKA-Brownish
Oldsmobile Station Wagon). Even though this is not his vehicle of choice, Adam has made the best out of an embarrassing car and uses it to
expand his sports storage space. Surfboards, in-line skates, tennis rackets, basketballs, tennis balls, volleyballs, caps, jackets and
occasionally you will find a book or two underneath all the clutter. All children are a mixture of their parents. However, Adam has taken this
concept to a higher level. Adam's beautiful wavy black hair (thank God it grew back after he shaved it off), is like two distinct puzzle pieces
that fit together perfectly. (Did I say Adam was a puzzle? Must be a Freudian slip.) The crown of Adam's ebony hair is the same golden blonde
color as his mother's hair. He got it the same way Mom did-out of a bottle. Now when people ask whom Adam looks like, we can legitimately say,
both of us.   (He took his passport picture with his hair like this and he will have to live with this image for 10 years. Of course, he can pay
for a new passport, but Mom ain't gonna do it. Now, I need you to picture Adam (all 6 feet 2 inches of him) laying in a bed that is almost too
small for him. He has just finished a high school volleyball tournament in which he played 8 games and then he played a basketball game with his
friends. He is covered with a thick, heavy comforter that is pulled up to his neck and a black ski cap covers his head. (It gets pretty cold in
California in winter. It gets down to about 50 degrees.) Adam is reading his history book with Lucy, the cat, curled up at his feet and both are
listening to the melodic tones of classical music. I guess it is nice to know I have kids who do what they like to do, even if it is not
following the crowd. Except for his hair!!!!!.

Jasmine is 12 going on 27 and sadly she knows more than anyone does. (Except maybe Adam because he can argue longer and louder than she can.)
Jazz can be likened to the cartoon character "Pigpen" with an ATTITUDE. I know where Jasmine is and has been at all times. There is a trail of
books, jackets, shoes, socks, plates, glasses, paper wrappers, cream cheese containers, pizza crusts and half filled soda cans. Of course, she
will heartily argue that she didn't make the mess. (But there was no one else at home.) And she is very upset, that we accuse her of being
messy. We have an intercom in the house that she conveniently turns off so she doesn't have to hear us calling her-especially before and after
dinner. When she is asked why she didn't answer. She will respond that she didn't hear us. We don't believe her and say, "But we called you four
times." "No you didn't, you only called twice." BINGO!! She heard us, she just ignored us. ATTITUDE. Jasmine's artwork is still spectacular. (Is
it a requirement that artists have an attitude?) She is following in her sister's footsteps and plays soccer very well. (When she wants to.)
She has been wearing braces for a year now and the orthodontist is very pleased because she listens very well to him and is making great
progress. (Note: Talk to orthodontist and find out how he makes her listen and do so well.) In those moments when she is just Jasmine, she is
sweet and fun. My mother told my sisters when I was 12 that I was just going through a phase. Mommy, how long did it take me to get out of my
phase? (Be nice!)

Life is meant to be lived and enjoyed. The very sad part of life is that we cannot live forever. This summer I lost my 92-year-old father. It is
not something that I have come to terms with yet. I do know that what he is most remembered for is his sense of humor. That is a gift that Daddy
passed onto me. So whenever someone makes you smile, think of Daddy and how he made everyone around him smile.

I hope your New Year will find you healthy and will be filled with happiness and laughter. Remember, your laughter will be a gift from Daddy.

Love,

Nasr, Virginia, Amira, Adam, Jasmine, Butterscotch, the dog, Lucy, the cat, \& Cleo, the new, obnoxious kitty.


%\end{document}
