%\documentstyle[11pt]{article}
%\setlength{\topmargin}{-.75in} \setlength{\oddsidemargin}{-.3in} \setlength{\evensidemargin}{-.3in} \setlength{\textwidth}{7in}
%\setlength{\textheight}{9.75in} \setlength{\parindent}{0.2in} \setlength{\parskip}{0.1in}

%\begin{document}

Happy Holidays!                                                  \hfill December 2002
\vspace{.5in}
As I begin to write this letter, the world is in a troubled time and I wonder if the world is ever not troubled.  So it is during this holiday
season that I hope I can add some cheer and laughter to our daily lives.

Nasr started travelling again this year.  He travelled to London and Egypt with Ami and then later to Switzerland. Most would be very jealous of
his schedule.  He says he is working, but he never leaves the house until 10 a.m.  He claims this is because he is trying to beat the traffic.
His self-proclaimed title is "big time manager in charge of research," but for the rest of us, he's an absent-minded professor. On the days he
works at home, he exercises, and then enjoys lunch and coffee out.  He believes these 3 hours out are the same 3 hours he would spend in
traffic.  (Who believes that?) Nasr loves to make things last-particularly his cars.  This year we had an option of buying a car for Jazz or for
Nasr.  Of course, he was going to be a martyr and buy one for Jazz.  I stepped in and made some noise.  He always buys for others before
himself.  I quietly (ok, not so quietly) convinced him that he needed to buy a new car.  He consulted his car expert friends and thought and
thought and thought some more. Finally, one day we saw an ad in the paper, drove to the dealer, and then drove home in a new Mercedes.  Nasr
loves his "glider" as he calls it.  Now he wonders why he always gave the kids new cars and never bought one for himself.   We are moving into
the ultra modern era of computers being networked. We both sit in the same room with Nasr on the couch and Virginia at the desk, but we don't
have to use our voices. We merely email messages back and forth to each other.  So many people have complained that the art of letter writing is
gone. The Ghoniems are back to letter writing, however, we just might be losing the art of conversation.

Virginia who was blessed with being born into the best, most supportive, family in the world lost her car at the same moment that she was
writing this letter.  (No, not her marbles, her car.) A teenager was driving much too fast down the street and my poor Acura took the brunt of
the crash. At least Nasr was home.  However, when Nasr travels, the animals are in the house to keep me company. (Shh-don't tell Nasr.) When the
kids were little, I never had a free moment.  I couldn't even use the restroom without company.  Well, the animals have the same craving to keep
"Mom" company.  The two cats are usually fighting-one can open the door, so she comes in to complain, then, the other cat pounces on the first.
Meanwhile, the dog mills about my feet, silently begging me to make them stop fighting.  It's great to be loved, but there is a reason why kids
grow up and hopefully move out.  Nasr was on a business trip in Switzerland. (Why do things always happen when he travels?) I had just finished
a grueling 6-hour drive from Stanford.  It was midnight and I merely wanted to sleep because the Monday morning alarm would sound at 5 a.m.  As
I walked up the steps, Jazz asked me, "Why are there two holes in the wall next to Adam's door?  Too tired to do anything but shake my head, I
walked into Adam's room and calmly asked, "Why?" His reply was simple, "I got mad." It surprised me how calmly I responded, "Next time don't
take it out on the wall and call someone to get it fixed before your dad gets back from Switzerland."  As the week went on, I gave Adam gentle
reminders and then not so gentle reminders to get it fixed-if you have teenagers you can relate. Suddenly it was Saturday morning, Nasr was
coming back on Sunday, and I still had two giant holes in my wall.  Saturdays are my day at the beauty salon. The manicurist, who has a pulse on
the world, was listening to me complain about my week, when I meekly asked if she knew of anyone who can fix dry wall. Her face lit up as she
exclaimed, " I sure do."  A few phone calls later and some deft maneuvering with the phone (I can't mess up my nails) I had a painter scheduled
to come to the house at 2 p.m. By that evening, no one would have guessed I'd had two six inch holes in the wall.  Adam came home just as the
painter was leaving and mumbled thanks. His problem was solved!  The moral of this story is: If you ever have a problem, don't write to Dear
Abby, visit your local beauty salon and all your cosmetic problems will be solved, hair, nails and dry wall.

Ami is almost 23, and is in graduate school at Stanford.  Unfortunately for our bank account, Ami still has the travel bug.  She went to Egypt
this summer with her Dad and enjoyed spending time with her friends.  (Ami would go to Egypt every 6 months as long as Mom and Dad foot the
bill.)  We never paid Ami for getting good grades, but somehow she convinces us that since she is doing so well in school, the appropriate
gesture is to pay for her travel.  I can usually agree, but this kid wants to travel more than Gulliver.  She hasn't been to Colorado to visit
her Uncle Steve, but she did meet up with him and his wife, Gayle, in London.  Ami knows London well and became a tour guide. Ami has a new beau
who lives in Santa Monica.  Don't forget Ami's school is in Palo Alto.  Long distance relationships create a very healthy economy.  The money
they spend on long distance phone calls, airline tickets and gas-especially along Highway 5 will keep the California economy strong.  Maybe I
should suggest this idea to Mr. Bush-his other ideas are just as harebrained! (Oops, got a little political.)  Ami will be completing her first
Masters next quarter.  Then, she will continue working on her second Masters.  If you think Ami enjoys being a perpetual student because she
wants to travel, you may be right. However, the bigger problem is that Ami has a fear of work. Let me explain, she is not afraid of hard work,
she is afraid of not having enough time to play.  However, being a good mother, I knew I had to help her get rid of her fear.  Therefore, I
lied.  I told her that work is fun, almost like playing.  I also said gainfully employed people get to take LONG two-week vacations and after a
few years they get an ENDLESS three weeks off.  Unfortunately, she caught on and still doesn't want to work.  But, if she stays a perpetual
student at Stanford, Nasr and I will be in an economic crisis.

Adam is 20 � and is gainfully employed. He is working in the bio-medical field as a lab technician.  Adam has come to the conclusion that money
is the root of all evil-it is especially evil when you don't have enough. (Anyone else ever have the same thought?)  Adam, also, is a great
dancer.  He shows us his moves that look more like acrobatics and laughs at his dad's cool, barely moving, minimalist routine.  I tell Adam that
we used to win dance contests and claim his dancing ability comes from his mom.  He smiles and pats me on the head and thinks, "Mom, you've
become senile in your old age."  Adam has taken up a new hobby-learning to be a disc jockey. He has 2 turntables, equalizers, woofers and
sub-woofers and plays loud, pounding music with intermittent scratching sounds.  The kids say it's music, I have another term for it.  Adam, is
coming to the realization that he is becoming more like his father.  On one level he thinks this is good because "you want to be an intelligent
successful guy like dad." (I wish Nasr had been here to hear that instead of "working" in Washington D.C. It just isn't the same hearing a
compliment second hand.) On the other hand, Adam's very nervous that he uses the same terminology as his dad.  One evening, Adam, his girlfriend
and I went to rent a movie.  Adam was busy complaining that he had to work hard all week and that he was tired. (Those are Nasr's exact words
every Friday night.)  Adam said he hoped he could stay awake during the entire movie.  I laughed and told them that Nasr has the same problem.
Then, I explained that Nasr's reason that he sleeps during the movie is that he is so relaxed around me, he just can't stay awake.  Adam's
girlfriend's eyes grew very wide.  "No way!" she exclaimed. "That is exactly what Adam tells me when he falls asleep during a movie."   Is this
a genetic issue-like father like son?  Or is it a gender thing and women just have more stamina than men do?   Let me know your opinions.

Jasmine is driving and driving me nuts as only a 16-year-old can.  Jasmine has always been into make-up. You should see her huge two feet long
make up case---I'm not exaggerating---that is stuffed with make-up.  If she has all this, can you please tell me why, why, why does she need to
go into my meager stash of make-up, use it up or take it into her bathroom?  Then at 6:00 a.m. when I'm trying to make myself look human for
work, I have to go searching for my makeup or pound on her door, screaming over the throbbing bass of her radio, "Where's my makeup?"  Nasr
rolls his eyes, covers his ears and says I should lock my makeup in the safe and take it out every morning so he doesn't have to listen to me
getting so upset. Ladies, do you want to go to the safe to get your make up? Yes, I will continue to rant and rave until she leaves mine alone.
Now, that said, Jasmine is a very good makeup artist.  Whenever she does my makeup, people keep looking at my face and tell me I look really
good. (Maybe she makes me look like someone else, but at least I look good.)  If you want your makeup done, talk to Jazz, but please plan on it
taking a LONG time.  I usually spend 5 minutes (I know it's obvious), but she will take 40 minutes or more.  Like many artists, Jazz is a
perfectionist when it comes to her art.  Oh, if you are a member of my generation, be prepared for her to complain about your droopy, gooey,
rubbery, eyelids that move when she applies the liner. (It's tough to take the abuse but definitely worth it.)  I get my revenge because I've
discovered what really grosses Jazz out.  Nasr and I love to go for walks at night and quite often we invite Jazz to go along.  However, she has
specific rules for us.  We may not hold hands while we are walking, and oooohh, don't ever think about kissing each other.  So, if you do catch
a glimpse of us, Jazz is usually walking ahead of us and we sneak our hand holding when she's not looking.  Of course, when we are feeling
particularly ornery, we hold hands right in front of her, just out of pure parental spite and gleefully watch her squirm.

This past summer my dear sister, Susie, succumbed to ovarian cancer.  No one in my family can understand why Susie had to die so young.  I've
come to the conclusion that God needed another angel.  Next time you meet a quiet, creative, cat loving, person with a beautiful smile who finds
it easy to laugh, think of Susie.  And this holiday season when you hear a bell ringing, it IS Susie getting her wings.

Too often this holiday season I'm awakened by the pitter patter of little feet running across the room, then a hiss, a growl, a little more
pitter patter, the clicking of the blinds, another hiss and growl.  Nasr mumbles disgustedly,  "Why are the cats in here? You know I have
asthma. Can't YOU do something?" Suddenly, the radio clicks on to a melodic, sweet, voice singing, "There's no place like home for the
holidays."  I reach out, hit the snooze button, and throw the pillow over my head to drown out the growling and think, "That's the truth!"

This year has been filled with lots of laughter and far too many tears.  We all pray for peace, happiness, good health and long life for all our
dear friends and family.

%\end{document}
