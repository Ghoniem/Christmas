%\documentstyle[11pt]{article}
%\setlength{\topmargin}{-.75in} \setlength{\oddsidemargin}{-.3in} \setlength{\evensidemargin}{-.3in} \setlength{\textwidth}{7in}
%\setlength{\textheight}{9.75in} \setlength{\parindent}{0.2in} \setlength{\parskip}{0.1in}

%\begin{document}

SEASONS GREETINGS\hfill December 1987\\


Another year has gone by and I must say our lives have been enriched, new horizons have been sighted and my patience has been s-t-r-e-t-c-h-e-d!
Again, I'm apologizing in advance for a typed letter. I wish I had the time to write a letter this long to everyone, but as there are so many
people I want to share our lives with, it is totally impossible to be completely personal. I'd have to start in July to finish by December, and
then what I'd written in July would be old hat, so I'd have to rewrite, but there' d be no time so...AAAAAHHHHH...a vicious circle. Therefore
and thusly, you get this. The same holds true this year as last. I promise a personal rendition if you call us at (818) 891-1616.


NASR has had a full, busy year. He has adopted a new schedule. He leaves at 6:30 in the morning and comes home at 7:30 in the evening. He uses
the excuse that the LA traffic has gotten so horrific that if he doesn't leave early and return late, he'll spend four hours commuting only 16
miles. You can believe that if you want to, but as you read on, you may come to a different conclusion. Nasr also spent five, lovely,
fun-filled, stimulating yet relaxing weeks in Europe and Egypt. He spent two weeks in the French Alps, hiking and sightseeing. He says he went
there for a workshop. He then spent two weeks relaxing with his family in Egypt, swimming in the Mediterranean, taking walks along the Nile and
enjoying his family's company. He spent the last week of his trip in southern Germany, meeting old friends, sightseeing in the museums and the
zoo and shopping. Of course, I won't complain all that much about the shopping as he was shopping for me and the kids. Again, this was a
"conference". He came back feeling very good about the conference and workshop, and said he gained a lot from them. My question is, however, was
what he gained from it really work related?


VIRGINIA, while Nasr spent his time in Europe spent five, 1-o-n-g, lonely, frustrating, exasperating, and earthshaking (we had two earthquakes
while he was gone) weeks in Los Angeles with three very active children. (That's too long a story to relate here.) As a family, however, we did
take a wonderful vacation in Wisconsin this summer. I got a chance to see my high school friends at our own personal 15 year reunion. (I
graduated when I was 6). It's nice to know that time doesn't change what's deep inside a person, even though the mirror tells us differently and
our kids nearly bust themselves trying not to laugh at our high school pictures. I also had the honor of organizing the family ancestral photos.
Sounds impressive? Actually, Mom had several boxes filled, helter skelter with pictures of total strangers, or so I thought. These pictures were
of family, grandparents, cousins and friends. Poor Mom, I think she regretted letting me do the job because I wasn't content with just pasting
them into the book. I wanted to know who all those people were and how they were related to me. I know I taxed her memory and we spent lots of
hours on it, but I thoroughly enjoyed it. The only thing I regret is that I couldn't find a picture of Daddy as a young man dressed in something
other than overalls. I'm sure such a picture exists and my next crusade is to find a picture of Daddy, with hair and dressed in a suit. When I'm
not vacationing, I'm busy writing. I'm determined that someday I'll publish a book. So, hang onto this letter. It is an original Virginia
Ghoniem letter and perhaps someday it will be worth something. At least it will help start a fire in your fireplace.


AMIRA, 7, is doing well in third grade. I admit I was a little worried about having her skip second grade, but it hasn't affected her at all,
and she even was citizen of the month in November. Amira is not content with mere studies. She needs other activities to keep her busy and not
bored. In ballet, she is doing beautifully. She is graceful and elegant, and can move her body in ways that Mom can only dream about doing. She
gave a piano recital at the beginning of November. She played Beethoven's "Fur Elise" and she did it wonderfully. She was very poised and calm
while playing on the Grand piano. Her one little quirk is that she loves to read while standing on her head and practicing her gymnastics. She
got the love of reading from her mother, but I assure you, that the standing on the head part must have come from Dad.

ADAM,5, is no longer juicing Jasmine's head, but he has devised new devious ways of irritating her and getting some negative attention. Jasmine
will "accidentally" fall according to Adam. But, mother's trained eye has seen him stick his foot out and trip the teetering toddler. And of
course, he thinks a simple "I'm sorry" is going to forgive all his sins. He is seriously thinking of making a tape that only says "I'm sorry" on
it. He'll keep it on all the time so then he won't have to be bothered with unimportant apologies that have no meaning to him. Yes, he's smart
and he's doing very well in kindergarten, too. He wishes he could spend all day in school. (Should I say it? No, you understand.) Adam, in
school, is a wonderfully behaved boy. He got the first Good Citizenship award in his class. Summer vacation at Grandma and Grandpa's was most
memorable for Adam. He loved helping Grandpa feed the cows in the barnyard. Adam, when he is excited, talks and walks, and doesn't watch where
he puts his feet. Yes, you're right. He backed right off the wagon and landed in a pile of manure. Adam popped up sputtering and accusing
Grandpa of pushing him off the wagon, and to this day, Grandpa still can't stop chuckling long enough to confirm or deny it.

JASMINE is 15 months. I'm reluctant to admit this, but I gave birth to a monkey. No, don't get me wrong. She is a beautiful little girl with
bright, shining, blue eyes and a head of curly, blonde hair. But somewhere, way, way back, one of her ancestors must have made a wrong turn in
the evolutionary chain. (Of course, I'm not naming names, but we know who grew up on the African continent, don't we.) Believe me, I'm not
making rash statements, I can back up my claims. I will put Jasmine in the family room and walk into the kitchen. By the time I turn around to
glance at her, she's standing in the middle of the table reaching for the chandelier, just wishing she was an inch or two taller so she can
swing on it. And, if she's not on the table, then she's standing on top of the television or doing a balancing act on the fireplace or running
across the back of the couch. Feeding time is strenuous exercise. She doesn't sit calmly in her high chair. (Don't tell me to tie her in because
she can get out of places that would have baffled Houdini!) She'll sit calmly in the high chair for 30 seconds and then stands up and starts
doing deep knee bends while I'm trying to stuff food into her mouth. She's a smart monkey, too. As soon as Adam comes near her, she lets out a
warning screech that would rival any of those primates in the zoo. Adam, then walks up to me and says, "I don't think Jasmine likes me. What do
you think?" Should I lie? 1987 has been filled with a wide variety of happenings and lots of happiness. I can say it now because those antics
are in the past and I thank God I'm not a psychic because I don't want to know what the kids have in store for me in the future.

We hope all of you have had a great year and we pray that next year will bring you even greater happiness and joy.



%\end{document}
