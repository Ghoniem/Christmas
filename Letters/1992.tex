%\documentstyle[11pt]{article}
%\setlength{\topmargin}{-.75in} \setlength{\oddsidemargin}{-.3in} \setlength{\evensidemargin}{-.3in} \setlength{\textwidth}{7in}
%\setlength{\textheight}{9.75in} \setlength{\parindent}{0.2in} \setlength{\parskip}{0.1in}

%\begin{document}

SEASON'S GREETINGS! \hfill December 1992\\


1992 has been a year to forget! It has been filled with trials and tribulations too numerous to recount and too boring to read about. Just
please pray that 1993 will be a little easier and less painful. But I'm here again to remind you of the antics of the Ghoniems and to let you
know we are still here and still in one piece. (We only have a few screws loose.) It's surprising that we are still intact considering the
feeding frenzy that the children have when they come home from school. They will devour anything that is fried, fattening or fun and doesn't
move. So there's no standing still in this house.

Nasr, has now reached the age that he's forgotten it. (However, his children do love to remind him.) Nasr is working hard at the University and
is also working on his consulting business. It keeps him very busy. This year Nasr has discovered how much fun a NEW computer can be. He has
spent endless hours getting the right programs and the systems adjusted perfectly. He also has played musical computer. He had a difficult time
figuring out where he wanted to put it. First it was in my kitchen taking up MY desk space and MY kitchen drawers. Then it was moved to a bigger
desk, which took up too much space in the nook. Now it is up in the study, which is great, but soon Nasr is getting a portable computer with
MORE POWER than this mega computer. I'm worried that soon the Ghoniem household will be taken over by computers and I'll wonder whatever
happened to Nasr. This year Nasr did some more travelling. Virginia stayed home!!! He went to Egypt to visit the condo we bought on the
Mediterranean, and then played in Brussels with some good friends. He said he went there to work. What do you think his reason was for going?

Virginia, has not forgotten her age, she just chooses to ignore it, and prays it will start going backwards. Yes, Virginia, there is a Santa and
he is going to give you a chauffeur driven limo. Well, if we are going to believe in fantasies, I might as well go for the gold. Actually, I'm
the chauffeur and I'm still driving Mom's taxi-the station wagon. I have convinced myself this is a luxury mobile because IT'S PAID FOR! We went
in search of a status mobile and soon realized we would have to pay $600 a month, plus $10,000 down. Hell, (oops I forgot this was a Christmas
letter) I mean Heck, my 1986 station wagon just grew in status in my eyes, especially considering that the school district has taken 12\% of my
salary and is threatening to take more. The Union threatens STRIKE (like I said, pray 1993 will be a better year). I've started a new endeavor
to help offset the loss of salary. I think it is great and I'll be talking more about it in future letters. Teaching is absolutely wonderful
this year. I'm teaching a second-third grade combination in English, even though there are 7 different languages spoken in the classroom. I
discovered that too many children cannot read the books that are written for 2nd and 3rd graders, so I've done some writing this year and
hopefully, somebody will like what I've done. Don't hold your breath because too many writers have keeled over from expectations. When Nasr was
travelling, I had a visitor. The rain in California has produced an abundance of rodents. Yes, you guessed it, I found a small rodent (I don't
want to say rat because that gives me the willy nillies) in my Jacuzzi BATHTUB! AAAAHHHHH!!!! You better believe I screamed, and screamed and
screamed. Nasr was playing in Belgium, he didn't hear me. My children were asleep, they didn't even roll over. Not ONE of the neighbors called
to find out if I was being murdered. (That was really scary. I knew I was all alone and had to deal with this creature myself. Meaning who could
I convince into helping. One neighbor just had heart surgery. Was this serious enough to get him out of his bed and save me? YES! But, I figured
he might look at it differently. Another neighbor works nights about 60 miles away. I realized he might find it hard to come and help during one
of his breaks. I only had one left. I was desperate. I called and found myself talking to his wife. I asked if her husband would be brave enough
to save me from a creature. She laughed and said he couldn't save me, but she would. This wonderful, brave women showed up at my house minutes
later, I never left the bathroom, I didn't want it to escape into my bedroom. She was armed with a net, and a pillowcase. She bravely put the
case over the mouse, scooped it up in a net and whisked it away in her van. Now do you know why I hate it when Nasr travels? He gets to play in
the Mediterranean. I get rats in my tub!

Ami, is only 25 days away from 13. HELP!! I know, I better enjoy these last few days before I have a real teenager in the house. I'm hoping that
somehow the teenage flaw is going to skip a generation and my children will not be inflicted with that monstrous disease "TEENAGE-ITUS". I can
hear every parent of a teenager now. They are either thinking up ways to insure that my children get this disease because its not fair if I'm
spared. Or, they are laughing hysterically at my naivete. Ami is doing superbly in eighth grade. I couldn't ask for better grades. She is a
beautiful, self-motivated, young lady. (So, I'm a little prejudice, it's a mother's prerogative.) Ami can save money very easily. She has a
spendable account in the bank (not a college fund) and she is counting the days until she can have her own ATM card. This is true power, a sign
of maturity, and a headache when you have your own ATM card. Hey, I can write her checks instead of paying her in cash, which I never have. Ami
(no, I didn't get another daughter, she just doesn't want to be called Amira) is still a kid at heart. She loves dress up. But no, she doesn't
go through old clothes that are stuffed in the toy box. She dresses up in MY clothes from my closet. (They look better on her, too.) It is not
simple dress up. It is high tech dress up. She gets out the video camera, puts on a show, with several costume changes and then shows the
neighborhood her production. It's not that I'm not proud of my daughter's budding film endeavors, it's just that when she is finished my closet
looks like a hurricane hit. Then, when it's time to clean up she suddenly loses all energy, pep and personality. She turns into a whiny, tired
teenager. Yuck! Guess who has to clean up? Yes, you're right. Ami does it.!!

Adam, is a handsome, energetic 10 year old. Adam is in the 5th grade at San Jose Highly Gifted Magnet. He is very ecologically minded. He always
tells me when I should be recycling or when I'm doing something I'm not supposed to do. (Isn't it supposed to be the other way around? ) Well,
in many instances I do tell him what to do, especially, when I want him to help around the house. He wants to earn money, but he figures we
should give him money just because he EXISTS. Now, this is a difficult concept for Nasr and I to understand. We have always thought that you get
money if you work for it, but Adam definitely does not ascribe to that way of thinking. When I do try to tell him what to do, because of our
different philosophies on the need for work, the discussion often gets heated and a quite LOUD!. Why is it that when I tell him what he should
be doing, he doesn't seem to accept it as well as I accept his little suggestions on saving the environment? Do you think it would help if I
explained to him that there is such a thing as noise pollution? You're right. It would only add to the unnecessary hot air in the environment
and piles of other unmentionables.

When we do convince Adam that he must EARN his money, he does so very grudgingly. But once he has that money in his pocket, it immediately
starts to burn a hole. I think he must coat his money with an environmentally safe acid, because I've never known anyone who has to get money
out of his pocket , and into someone else's hands faster than Adam. Adam and Jasmine have a running feud, but at the same time they are
inseparable. The other day, Jasmine was provoking Adam and I must say Adam used a great deal of self-control, because that is one thing Adam is
NOT known for. He turned to Jasmine with his dark brown eyes blazing and said, "When I'm awake and tired, I'm potentially dangerous." Jasmine
was not astute enough to recognize the threat, but when a ten year old lets you know that he is potentially dangerous, look out! Is this a
statement of things to come? (I still think kids need to come with users manuals.) Finally, when you play marbles with Adam, be careful. He
plays by his own set of rules. Also, watch those hands!

Jasmine, is six and is in first grade. This little girl definitely takes after her mother in some ways. She is a true tomboy. She spends more
time with the boys than with the girls. She can rough and tumble with the best of them. (Just like dear old Mom!) This year Jasmine has fallen
in love with the Dalmatians. Almost every article of clothing she owns has a dalmatian on it. She must have 50 dalmatian items or more. I do not
think that it would be too much to ask that Disney sent us a personal Christmas card this year as Jazz has helped support them and has added to
their profits heavily. Now, I wouldn't want anyone to call me a meddling mother, but Jazz is going to be a top notch business PERSON when she
grows up. Jazz is the exact opposite of Adam when it comes to money. She has no desire to spend any of her money. However, the one similarity
she does have with Adam is his philosophy about working for money. She is somewhat easier to convince she must clean up her room to earn her
money. She does not earn as much as the other two, but she tries to make up for it. If we tell her she will get \$2 a week for cleaning her
room, she will try to convince her dad to give her \$2 a DAY for cleaning her room. If anyone ever needs money, Jasmine is the one to go to.
During this Christmas season, when I watch Jazz counting her money I am reminded of Scrooge. Several times a day this child counts her money.
And she doesn't like \$20 bills. It doesn't look like enough money. She wants everything in ones. She uses several wallets and keeps asking for
more because the ones she has are too small. The other day she came up to me and asked me quite innocently, "Mom, can a five or six year old kid
start to be a millionaire?" It took a few moments to pick my jaw off the floor, but I smiled and said, "Of course!" If my child wants to be a
millionaire, I'm going to be the last person to discourage her, especially, when I can think of a few things I'd like to get that only my
millionaire daughter can get for me. So, if we have any long lost relatives out there who are looking to give a poor, put potentially successful
young lady a jump start, I'm sure she'd be willing to set up a meeting with you and BARGAIN a lot of your money out of you. By the way Mom, Jazz
has big plans for your oil wells once they come in.

The Ghoniems may not be millionaires, but we do have many riches. The antics of our children fill us with happiness (even when we'd like to
scream) , and Nasr and I have wonderful times together (when we see each other). It's our family and friends that make us wealthy beyond
compare. It's the fun, silly, sad and thoughtful times that we spend together that makes us real millionaires. I hope 1993 will bring everyone
peace, happiness and lots of laughter.

%\end{document}
