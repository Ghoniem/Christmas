%\documentstyle[11pt]{article}
%\setlength{\topmargin}{-.75in} \setlength{\oddsidemargin}{-.3in} \setlength{\evensidemargin}{-.3in} \setlength{\textwidth}{7in}
%\setlength{\textheight}{9.75in} \setlength{\parindent}{0.2in} \setlength{\parskip}{0.1in}

%\begin{document}

Happy Holidays\hfill December 1995

December 1995 Happy Holidays, The mood is set with the chandelier twinkling overhead, carols playing softly in the background, the children
(including Nasr) are all nestled and sleeping in their beds and Mom is scurrying around frantically finding boxes for gifts, wrapping gifts,
untwisting the tangled tape from her fingers, and in general going nuts from the holiday madness. When people are asked what is their favorite
holiday, they often reply that it's Christmas. My answer is the same. Yet, it makes me wonder with how crazy I am at this time of year, why do I
like it so much?  Now that I have the first round of gifts wrapped, it's time for me to get my letter written. The only problem is that it's a
few minutes after midnight. You got a better time to write a letter?


Nasr has spent the year rebuilding the house, AGAIN! Or shall I say, STILL. All I know is that we have been working on this house since the
minor earth movement in 1994. Nasr says we'll be done sometime next year. If I write about house repair in my letter next year, you might find
words that are inappropriate for this time of year. I'll believe that the house is done when I no longer have workmen traipsing through the
house. Nasr takes all the credit for designing the house. Any problems, it was the contractor's fault. Nasr is making sure that this house is
not going to swing, sway, jump or budge during the next earth movement. I was told that remodeling was difficult on a relationship. AMEN!!!
Remodelling can try the patience of a saint and well, Nasr has been an interesting creature during this time, but I wouldn't refer to him as
saintly. On top of the house, he is consulting (paying for the house, as in, no insurance), changing his areas of expertise at the University
and playing professor.   With all of Nasr's grey hair, people find it hard to believe that he's 21. Nasr's new toy this year is his sound
system. It has everything from subwoofers to digital surround sound. Now, Nasr's musical sound level is competing with his children.


Virginia has taken on a new responsibility at work. She is the coordinator, coordinating three federal programs at the school. It's a challenge,
but she's enjoying it. No matter how hectic it gets at work, it's still much more relaxing than the crazy schedule she keeps at home. Three
wonderful children, who have endless needs to be chauffeured constantly, to the ends of the earth and soccer and basketball games, put a lot of
stress on Mom. I have made an earth shattering scientific discovery. Once a mother reaches middle age (of course I'm not referring to myself)
her hearing becomes very acute. A mother can hear music at levels that teenagers could never hear. A mother is constantly telling the children
to turn down the music and the children are screaming that they can't hear it.   Another discovery is that husbands, once they have a mid-life
crisis and search for sound to surround them, their hearing reverts back to the low levels of a teenager. Yes, the noise in this house is
deafening and that only refers to the stereos, not the fighting, arguing and general "yuckiness" of having two teenagers in the house. The
highlight of my year was spending September in Wisconsin, helping Mom celebrate her 80th birthday. All the siblings were there. We're all alike,
so you know that Mom must have been glad when we went home, so she could have the peace and quiet that I crave.


Ami will be 16 in 29 days and behind the wheel of a car. Yes, she will be getting her driver's license. For Ami it is a sense of freedom. For
Mom and Dad, it's a sense of worry and fear for her safety. She is a good driver, but let's face it, this is Los Angeles. Her straight A's will
at least help with the insurance. To be completely honest however, I do look forward to Ami driving because she will alleviate some of my
chauffeuring time.  Ami is still living, breathing and playing soccer. She went to Hawaii this summer and her team won the "Hawaii Cup." Another
of her teams won the Division II AYSO sectionals. She's playing on the high school team, practicing for three hours a day. She is usually
running around outside or walking in the house with a soccer ball under her feet. The best picture I have of Ami is when she had a deflated
soccer ball on her head after soccer practice. The soccer ball was her soccer crown, but the cat eyed, rose colored sun glasses really depicted
what a soccer princess we have.

Adam at 13 is tall. Adam is taller than Mom. Adam is taller than Dad. Adam's head is so big from being the tallest, that he can't fit it through
the door. He did manage to cut it down to size however this summer. As soon as school got out, he decided to cut his hair. His father dutifully
took him to the local discount hair stylist. Adam described what he wanted and he got it, but he didn't like the style he wanted. He came home
and while Nasr was napping, Adam asked if he and his friend could fix it. Fix it has several meanings. The one meaning that Nasr was not
expecting was "shave it all off'. Nasr woke up from his nap to find the house covered in hair (most of the mess was in our bathroom), and Adam
was bald! I never thought Adam had big ears, but when you looked at him, all you saw were his Dumbo the Elephant ears. It happened that that
same evening he was invited to a black tie bar mitzvah. Now you must use your imagination when picturing my son. Remember, up until today, he
had long, black locks. Forget the locks, now his head is a pinkish black, about the same color as newborn piglets and about as scratchy. His
ears are sticking straight out from the sides. He has grown so much that the men's size small suit just barely fits him. He is wearing a light
blue suit, that he insists on buttoning. He is very thin and now Adam looks like a long, blue hotdog. He does not own an appropriate pair of
shoes for this formal affair, so he borrows a pair of wingtips from his dad. Wingtips make your feet look large. Now we have a long, blue
hotdog, with oversized clown feet, with a piglet colored head and elephant ears. His dad's large black shirt cuffs hang below the sleeves of the
suit and his tie is imprinted with bugs bunny dressed as a superhero. To give him a little class, he adds his glasses. I dropped Adam off and
prayed no one would recognize him as my son. Thank God his hair has grown back. I think his bad attitude about grades and studying was in the
old hair, because suddenly this year he realized that if you study, you get good grades and he's doing it.

Jasmine at 9 is still our horse loving artist.  She goes to the Conservatory of Fine Arts for formal art classes. The conservatory is at a local
college and Jazz is very proud of herself that she goes to college at age 9. Of course, this is a major commitment for Mom. Every Saturday for
five hours I'm chauffeuring and chaperoning Jasmine. Jazz is doing very well in school this year, but she does have a hard time getting up and
getting ready for school. One cold morning about 6:45 Jasmine called over the intercom. "I'm not going to wake up until someone conies up and
wakes me up."   Nasr dutifully went up and woke up his soundly sleeping daughter and she did get ready for school on time. I guess I really do
know why I like this time of year, even though it is hectic. We have a chance to reminisce about the year, talk with our friends and family,
share our gifts and good cheer. Why do I like the Christmas holidays the best?   Because Christmas is all about family and no matter how bizarre
they are, a family is the greatest treasure a person can have. I hope all of you are sharing and enjoying your families. (If I enjoy mine, you
gotta be having a good time!) Wishing you all good health and happiness for the coming year.   Let us know your news.

Love, Nasr, Virginia, Amira, Adam, Jasmine, Butterscotch, (dog ) and Lucy (cat) Ghoniem




%\end{document}
