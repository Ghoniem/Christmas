	Greetings of the Season!		\hfill						December 2012
\vspace{.5in}

	
It is a cold, rainy day in Southern California--ooh shiver!  The December temperature is a chilly 63 degrees and will drop to a super low of 54!  But there is hope, it should stop raining and warm up to a reasonable 77 degrees by Thursday.  With luck we may survive the cold snap.   So as I write my yearly note, the Christmas carols, the warm blanket and my hot chocolate will help me muddle through the frigid temperatures.

Nasr (aka Gidu) loves the beach and loves to bicycle.   For the longest time I thought Nasr loved to bicycle at the beach because of  the pounding surf, the fragrant, salty air, the huge expanse of the shimmering water going beyond the horizon to his far away memories of swimming and cavorting in the Mediterranean Sea.  Yet, I recently discovered otherwise. In case some of you are not aware, Beach Volleyball is a major part of the California beach scene.   As you glide past the players on your bike, you see their skills and great form in executing a cut shot.   Nasr, however, could not say much about a cut shot, but he does admire those bikini-clad forms.   Someone needs to tell Nasr he needs to keep his eyes on the road.  He has way too many close calls next to the nets.   Last winter when we were in Falls Church, Gidu and I had several opportunities to take Siena to the park.  Whenever Nasr leaves the house, he takes his iPad.  He makes sure he brings it just like I make sure I bring my purse.  So, at the park, Siena was swinging and she was watching her Gidu sitting on the bench working on his computer.  Siena looked at me shook her head and said, �What are we going to do with Gidu?"   Before I could respond, she shouted, "Wake up, Gidu.  You need to play."   He did wake up and started using the new Siri application on his iPad and iPhone. He is so happy that the application talks back to him. Nasr tells us it is great to talk to her, but he spends more time talking to her than he does the rest of us.  There can be three people in the room and he is still talking to Siri.  Should I be jealous?  He took Siri and travelled to Singapore again this year.  He, of course, enjoyed the collegial repartee that he has with other engineers and scientists, but his upgrade to the Presidential Suite in the hotel was the highlight of his trip.  Speaking of presidential, Nasr flew to New York with several of his friends to meet the current (as of December 1st) leader of Egypt, President Mohamed Morsi. However, before the visit, he had the chance to walk the very crowded streets of Manhattan. People speaking in every language of the world, busily heading to where they needed to be hurried past Nasr in all directions.  Luckily, he found the food trucks that sold Egyptian sandwiches.  He was amazed at the popularity of the food and flabbergasted by how much money the owners were making because the lines of hungry diners snaked for several blocks. Meeting the President of Egypt sounds super spectacular, but there were about 499 other people vying for the President's attention by shouting, arguing and praying.  Nasr sat back and watched the circus.  He decided he would leave politics to the politicians.

Virginia (aka Teta) packed for her summer trip.  I packed two suitcases filled with all the necessary items I would need for this year's adventure.  Then I carried the 15 feet down the hallway to the blue room. I unpacked and then waited anxiously for Amira and Matthew to come for Nasr's niece, Nina's wedding. As the family is growing, we figured it made more sense to put Amira and Matthew and the two girls in our bedroom--more room for all of them.  Even though I didn't travel very far, the summer was spectacular.  It was filled with family and friends and joy and laughter.  (We would have traveled, but the school year ended late and started early--so we couldn't take our usual trip.  (Don't worry, we are traveling to the Bahamas in December--think blue skies and warm weather.) Oh, the wedding was the highlight of the summer.  Elegant designer gowns, seeing family and friends, new and old, and a spectacular venue--all made a fun evening for all.  Summer didn�t end when I went back to work, so Nasr and I enjoyed the country club scene in the evenings and weekends.  We swam, worked out in the gym, and caroused at the Friday night dances.  The songs were familiar and just as it was when I was growing up; there were more women than men dancing. And just like then  the range of men's dancing ability went from the very adept swing dancers to the making a fool of yourself dancers.  They all were acting 21 or younger, but they were all members of the geriatric generation. The last song was the Achy Breaky Heart.  I knew that the next morning, the majority of dancers would be singing and humming the same tune, but the words would be Achy Breaky knees, hips and muscles, not hearts. I had the chance to play a part in a play this summer.  The play was an immigrant story and the immigrants happened to be Egyptian.  The role I played was such a stretch, that you need to be impressed.  I played the role of the American married to an Egyptian professor from UCLA.   My siblings have called me a drama queen for a long time.  Do you wonder where Jazz gets her drama from?  I worked with a professional director, professional actors and many volunteer actors like myself.  We had a great time.  I learned so much about acting, the theatre and myself, in particular that I still have the ability to memorize the monologue that the writer so aptly and beautifully wrote. Now, retirement looks even better since the acting bug bit. (Don't look for me in the local theatres, yet!) Nasr loves technology and I love my cars--I don't have a clue how they run, but I love the way they look and feel.  Nasr gave me an early Christmas gift--a Mercedes E-350 luxury vehicle. Oh, it is like riding in a bubble. When you see me floating gracefully down the highway, don't bust my bubble.

Amira and Matthew brought a new life into the world at the end of the last year.  Jada Alexandra Koerner was born on December 19, 2011.  Teta got the chance to spend time with her during her first three weeks of life.   Teta very happily took the late night shift to give them a little break.  It is a good thing that LA is three hours earlier.  3 a.m. is only midnight for Teta, but way too early for Mommy and Daddy to get up.  At the time, Siena, our little three year old queen, was not thrilled with her little sister.  Since the baby arrived just a day after Teta came in, Siena assumed Teta had brought Jada into the house.  Siena's issue was that Jada was taking Mom's time away from Siena.  Those three weeks passed too quickly and soon the family was running smoothly on their own.  Siena loves cats (we tried all the remedies that you sent last year, thank you, but--they don't work) and since she can't have a real kitty, she wears them on her clothes.  Thank goodness for the Hello Kitty fashion line.  In early April, Siena donned her bright pink rain boots just before crawling into bed.  Amira tried to convince her that boots were not bedtime attire; Siena was not convinced.   As a good mother, Amira picks her battles and let Siena sleep with her boots on.  A short time later when Amira checked on Siena, she was sleeping on the heavy boots.  Siena is as much into fashion as her Aunt Jazzy.  Years ago, when Jazz was two, I predicted she would be a clothes fashionista.  Will Siena love clothes and fashion as much as her aunt?   My children promised they would pass on the information to you if Siena takes after Jazz. Anyone taking bets on this?  They stayed for a week, but it went by much too quickly.   Gidu and I dropped them off at the airport.  While I was holding Jada, I tearfully said good-bye to Siena and hugged her.  Siena, looking relieved, said "Good Bye, Jada."    Needless to say, Siena was utterly disappointed when I gave Jada to Dad and kissed her good bye, too.  Jada had the chance to shine at the onset of her 11th month. She took her first real steps on Thanksgiving Day!   Amira and Matthew ran a 10-mile marathon in the fall. This will keep up their stamina. They will need it for when the girls are teenagers!

Adam and Oanh surprised us.  We are going to be Grandparents again!!!!!   Their little one is due in February.  Now, it was a surprise for us, but they planned for a child to be born in the year of the Snake.   Adam is the year of the Dog, and Oanh is the year of the Rooster.  A child born in the year of the Snake will be the most compatible with them.  This is not a new art, just one that is new to me.  In a few moments of weakness--ok, in moments of utter frustration, I wonder if I should have planned to have children who are compatible with me.  But, then I realize that life would not have been so fulfilling and I would not have developed such a wealth of patience.  When I pried Adam for a hint about a name, he looked at me pitifully and said, "Mom, don't even ask. You won't know until the baby was born."  I know I can wait because of that patience that I learned so well.  Thanks to Adam and Oanh Nasr and I have participated in three 5k runs.  Ok, they are runs, but we walked.  Our first run was the Year of the Dragon Chinatown 5K run.  Since we had never walked around the area of Dodger Stadium, we sauntered through the streets stopping to take pictures and stopping to take in the panorama views of the Los Angeles.  What should have take a 45 minutes tops, took us a good hour and 15 minutes. As you have already figured out, we didn't win the race!.  But not being quitters, we tried our second 5K in Fullerton. The amount of food that you eat at the end of the runs negates any positive effect that the run gives you weight-wise.  But, the Superwoman, Superman feeling lasts a few days--just stay away from the scale.  The final run was in Long Beach at the Aquarium of the Pacific.    Nasr's sister, Susu, walked her first 5K.    Now that Oanh is very pregnant we are taking a little  "K" hiatus, but never fear, we will keep moving along.  10K next year???  Yeah, right!!  Adam is into antiques.  He acquired a 1911 Underwood typewriter.  Now, for him it is so unique with the ribbon and seeing the keys move. For us, it is just a little older version of what we learned to type on.  He wonders about the people who used the antiques. What were their lives like?  What were their dreams and did they accomplish them?  I can guarantee that no one will look at the 2011 Lenovo laptop on which I am composing this letter and wonder anything about it. Most likely it will be reassembled into an electronic robot for his grandkids.  Unless, Snakey (temporary name) will love computer parts, she will not get the same pleasure as Adam from her antiques.

Jasmine graduated from the University of California, San Diego in June in Human biology.  YEAH!!!!!!!!!! (How many exclamation points can I type to let you know how happy we are?)  Even though she graduated, she still had three lower division physics classes to take.  This summer she took two classes at UCLA and then the last she is finishing now. I hear a lot about velocity, magnetic field and torque.  Jasmine mentioned to her dad a stapler uses torque.  Maybe some of these principles she is learning she will be able to apply.  Every day he and Jazz lunch at the Faculty Center and they come home every evening describing the lovely three course lunch that they had which usually includes lamb, fish or steak. I compare my yogurt and apple lunch and think there is something wrong with this picture.   Jasmine is interning at the University of Irvine in Urology.   She is learning so much and tells us about the exciting research that is happening there. She is learning how to suture, consenting patients to studies, and observing surgeries. The most exciting gadget is the many-armed DaVinci robot that surgeons are learning to use.  In the future the doctor will not be standing over the patient. She will be sitting off to the side and the robot will be cutting, clamping and stitching as the doctor manipulates it remotely.  Science Fiction or Reality?  Will her Uncle's scalpel become an antique like the typewriter?  Better tell his children to hang onto one or two for posterity. Jasmine has been very generous to her mother this year.  I have been the lucky recipient of her hand-me-downs.  Close your mouths!  She is not giving me her clothes.  That would be both Science Fiction and ridiculous.  But, her two beautiful Chanel fragrances that she considers too old for her, are perfect scents for me.  Luckily, there is one size we do share in common--our feet size.  So, the brand name boots and shoes that no longer fit in her closet have come to me.  Now, Nasr has long complained about the number of shoes in my closet.  Jasmine's shoe closet FAR exceeds mine.  Particularly in the cost department!   She better get a good paying job to support this habit of hers.  Mom will not always able to buy these beautiful items.  But I must say, I do enjoy the hand-me-downs.  I wonder what they will look like when she buys them?  I can hardly wait!

There are so many things for which we all can hardly wait, but I hope everyone takes some time to  sit back and  enjoy the wonders of the season and takes pleasure in  the happiness of families and friends. I pray the New Year brings good health and continued happiness for you and yours.  Celebrate each moment, because once it is played, we can't rewind and get it back.   Happy New Year!!!
