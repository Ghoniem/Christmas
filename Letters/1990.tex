%\documentstyle[11pt]{article}
%\setlength{\topmargin}{-.75in} \setlength{\oddsidemargin}{-.3in} \setlength{\evensidemargin}{-.3in} \setlength{\textwidth}{7in}
%\setlength{\textheight}{9.75in} \setlength{\parindent}{0.2in} \setlength{\parskip}{0.1in}

%\begin{document}

SEASON'S GREETINGS! \hfill December 1990\\

Another year has passed and I'm not ready to write a Christmas letter. Give me another couple of months and maybe I'll be ready for Christmas.
But then again I just may procrastinate and end up writing, buying and wrapping in the last two weeks just like I am now. Maybe when technology
gets better I'll be sending you a video of the events of the year. Wouldn't that be great. I'd have a camera going all year long in each room of
the house. Then I'd just edit out the "good stuff and pass that along. Egads! Can you imagine how long it would take to edit a whole year of
five people's lives? I guess I'll stick to my ancient ways of using the word processor and laser printer.

NASR has been home most of the year and now I don't have much to complain about, nor do I get to describe those beautiful picturesque places
that he has visited. Starting in January he's taking a sabbatical so maybe next Christmas I'll describes scenes of beautiful downtown Santa
Barbara! Well, he can't travel the world every seven years. If he did, life would get pretty boring and what would we have to look forward to?
As of this summer, you can call my husband Farmer Nasr. He planted over 60 tomato plants and almost as many lettuce plants. We are still eating
tomatoes and will be for quite a while. There were too many lettuce plants even for this family of rabbits to eat and many of them went to seed.
But never fear, even after tilling everything under, the lettuce plants are coming back in full force and there are close to a hundred little
plants sprouting. If they survive the "winter" we may not have to buy lettuce for quite a while. Yes, Nasr has been working this year. He's been
busy consulting, researching and teaching. He's also been forced to help get the kids ready for school in the morning. At first it was pure
torture for him, but now he's got it down to an art. Actually, I think the kids like it best when Dad makes breakfast because he makes fun
things like hash browns, eggs, waffles, and special toast sandwiches. (Mom doesn't eat that wonderful stuff so there's no way that she'll make
it. But the rest of the family all think its great.)

VIRGINIA has spent the past year doing her stint of student teaching. Actually it is slave labor, but school boards and universities don't like
to call it that. Imagine putting in 8 to 10 hours a day for four months trying to control and teach 30 "wonderful" children and getting paid
zip! I can't complain because now its over and I'm officially a TEACHER. I never knew what a teacher really did until I worked as a substitute
and did my internship. It may be different in the midwest, but teachers just can't impart academic information, they have to be the be all to
end all to the children in the classroom. This year I'm teaching a classroom of Cambodian children. These children have seen more cruelties and
inhumanities to man than all of us combined have seen. They are not easy to teach, but when I do reach and teach them, it's a special moment.
When it comes to writing, I think my next book will be the wonderfully sad, funny and poignant moments in a classroom of young modern
immigrants.

AMI (AKA Amira) is an 11 year old young lady of surprising talents. I never imagined that Ami would be an athlete, but this year she has shown
what she can do. In the spring she competed in track and won several ribbons and medals. This fall she's out for soccer and their team is going
to be playing in the play-offs. Ami and her best friend, Rachel are inseparable. They look like Mutt and Jeff and if you find one of them,
you'll find them both. My little girl is now a sixth grader and she's desperately trying to teach Mom what is cool. I've learned it is not cool
to blow you nose because well, you know, someone might hear it. But the thing that ignorant, uncool Mom can't figure out is, why is it cool to
breathe through your mouth, sound like your dying and have "yuck" streaming out of your nose? If anyone can tell Mom why the first is uncool and
the second is cool, please write and explain. Everyone these days seems to have a wide variety of the different sizes and shapes of water
bottles. Ami is no exception. One minute Ami and Rachel have their matching water bottles, which are colorful baby bottles and are happily
drinking from them. The next minute she gets a call from one of the boys in her class. It's true, they grow up in a flash.

ADAM is 8 and is all rough and tumble boy. He's doing wonderfully in school and his ability to memorize is going to take him places. He told his
Dad he's going to go to school a few extra years so he can learn even more. (Dad cringed, because at one point he contemplated paying for the
childrens' education.) Adam is out for soccer also this year and he and his Dad have great games. I haven't decided who cheats more, but Adam
makes the most noise so I'm inclined to think it's Dad. Adam and Mom like to go hiking together, particularly in the winter. We've discovered
the park that has yellow prairie grass that looks as smooth as lions fur. Adam pointed out the grey brush that looks like the prickly hair of an
elephant. The rocks have faces, puppies and dragons carved by nature. With Adam's imagination he may be a writer like Mom because when we come
home from hiking Adam heads for the computer and starts writing about where we've been and what we've seen for extra credit in school. (Maybe
Adam will be writing his Christmas letter and describing all the scenic places he's been.)

JASMINE is four and going on about 37. I'll never worry about Jazz succeeding in this world. At UCLA when she was getting yogurt with her Dad
she wanted the chocolate Jimmy's. Dad said no. She screamed, ranted, raved and insisted. So to shut her up, the woman behind the counter gave
them to Jazz. She's learning early that squeaky wheels get oiled. But I'm afraid that one day, someone is going to grease her good! As a treat
for Nasr we were given several pigeons to barbecue. I was cleaning them when suddenly I realized that hanging onto the neck was a dangling head.
Ughhh. I screamed and informed Nasr if he was to eat pigeon he had to behead the birds. Jasmine was watching this whole scene and wouldn't allow
Nasr to throw the heads away. She tried licking the heads, but Mom held back a scream and told her we don't lick unattached heads. She had
several long conversations with these heads and then she decided that Butterscotch, our dog, should have the heads. But the dog is smarter than
I thought. She wouldn't have anything to do with those heads. This worried Jasmine so she asked if we should call the police because the
Butterscotch wouldn't eat the head. I calmly told her it really wasn't a matter for the police. She looked at me with those big blue eyes and
said. "Oh, then we'll have to call in the army!" I don't know how Jazz thought the army would help, but as the holiday seasons approach I hope
that the U.S doesn't have to call out its army and that this holiday season will be filled with peace. I also pray that this letter finds each
and every one of you in good health and brimming with happiness.




%\end{document}
