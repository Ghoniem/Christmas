
Happy Holidays!  \hfill December 2004
\vspace{.5in}
Again, I sit here listening to Christmas Carols while telling myself that I need to start writing the yearly letter.  I sat down to write the
letter over an hour ago, but was sidetracked by looking at the wedding pictures (http://www.bissonphotography.com/koerner). And that pretty much
sums up the year-THE WEDDING- but wow, what a wedding!

Nasr-the father of the bride-was a typical father. He tried to rein in the expenses by asking if we really needed this or that, but one look at
his daughter's pleading eyes and he melted and succumbed to spending more money.  I don't think all the wedding planning and commotion affected
him until about two weeks before the actual wedding.  You realize he took over 25 trips this year and the year isn't over, yet.   Do you think
one of the reasons he travelled so much was because he was avoiding the wedding planning?  You get to guess the answer for that one.  On top of
the wedding, Nasr is focused on the news from the Middle East.  This has been a year of dichotomies and the year of the "W".  When Nasr and I
walk, I speak of Amira and Weddings and Nasr talks of Adam and Weapons.

Virginia-this is the year she turned the "F" word.  Now, take your mind out of the gutter.  Think age!  I thought turning this large number
would be very difficult.  However, Nasr helped soften the blow by taking me on a 7-day Caribbean cruise with our dear friends. So, when the day
actually arrived, I was dancing the night away and swimming in the warm tropical seas.  I think every birthday should be met with such
exuberance \underline{\textbf{and a cruise}}.   Nasr is a little concerned about the cruise part of that statement.  I got a chance to travel
with Nasr on a couple of his trips this year.  After the cruise, we travelled to New Hampshire and then to Rio de Janeiro, Brazil.  You do
realize that 65\% of all gems are mined in Brazil.  Yes, I found a very special gem to bring home. Don't feel sorry for Nasr, you haven't seen
the women on the beaches of Brazil, and you know Nasr never passes up an opportunity to go to the beach!  Oh, on top everything else, I had gall
bladder surgery that had me laid up for about \underline{\textbf{6 hours}}.  Don't tell the doctor, he thought I actually recuperated.  But who
had time? I had to put on a wedding in four weeks!  Oh, and besides that, school was starting in two weeks!

Amira was married on September 19 to Matthew Allen Koerner on the grounds of the Doheny mansion in Beverly Hills.  Yes, it was every bit as
beautiful and romantic as it sounds.  Amira and Matthew worked very hard to create the most spectacular wedding and focused on every detail from
the invitations, programs, parasols, and homemade rose petal baskets to all the banquet items.  The stress level in the house was a little
high-especially as Amira was working full time up to two days before the wedding.  The most difficult part of her job was the 40-mile
commute-one way.  When traffic was bad, which happens about every�day, she would come home in a mood that bordered on hysteria.   (Matthew has
seen it, so this letter is not giving away any horrible secrets.) So, if Mom forgot to wait for the appropriate time for her to calm down before
asking her to make some decisions about the wedding,----ooooooh it really got ugly.   Planning a wedding is the most wonderful thing to do with
your daughter, but folks, it also is one of the most stressful things a mother and daughter need to do together.   I didn't want to make
decisions that I felt she should make (it was her wedding) and she didn't want to think about anything because she was too stressed.  As we were
returning from the bridal dress shop (a long, long drive and distance is equivalent to price), I asked Amira to get a piece of paper so she
could write down a list of things I needed to buy.  (It's not a good idea to drive and write at the same time.) I named a few items like ribbons
and boxes and walkie-talkies. Suddenly, Amira stopped writing, her eyes were wide with fear, and she looked at me like I sprouted antennas from
my head. "Mom, you are not�." I laughed and then calmly assured her that the walkie-talkies were for the school, not for the wedding.  She had
visions of her mother using the walkie-talkies during the ceremony.  (\textbf{I} wouldn't, but the coordinators made good use of theirs.)  A few
days before the wedding, I finally convinced Amira that I needed one hour of her time.  She kept putting me off because I think she was afraid I
was going to tell her she was an adopted child. Once we sat down together, I assured her that she wasn't adopted and gave her a book with
stories about my mother, her and myself.  We cried and laughed and grew a little closer.  By the time the newly weds made their grand entrance
into the reception behind the belly dancer sporting a candelabra on her head, we all knew that every little detail that seemed so unimportant at
the time, had created a wedding made for our princess, Amira.  Not only did the newlyweds dance to the Frank Sinatra's tune, The Best is Yet to
Come, but they also belly danced with young and old, Mid-Westerners and Middle Easterners. Now, that is a sight to see!!!

Adam was one of the groomsmen for the wedding.  Last Christmas he was in Marine boot camp and we were trying to find a date for the wedding when
we knew Adam was going to be home.   He finished boot camp in March, but he continued training in Oklahoma until the end of June.  He is well
trained to do things that a mother doesn't want to know about.  Adam enjoyed the training as soon as he was out of boot camp. He is a Lance
Corporal in Artillery, which means he knows how to shoot the BIG guns.  When Nasr and I went to visit him in Oklahoma, we had a chance to see
these very large weapons.  Adam was explaining that they practice for speed.  The reason they practice for speed is because once they shoot
their big 8-ton gun, which ejects a 100 lb projectile with a 30-mile range, they have three minutes to move it because the other side is going
to shoot back. That is something a mother doesn't want to hear.  Once he finished his training, several of his Marine friends came to the house
for a party. They were drinking and partying way into the night.  Suddenly, I was awakened at 3 a.m. by several inebriated men shouting. "Kill,
Kill, Kill." (That is something a mother doesn't want to hear or have the neighbors hear.)   While Adam was in training, his gear was stowed
neatly in his footlocker near his "rack" (bed for us non-military speaking people), which had to be kept spotless or he couldn't go to the
"mess" (food) hall. After he came home, he kept his room looking quite good.  But, when he moved into his apartment, he slept in a mess and made
a wreck of his bed. That is something a mother doesn't want to see.  Cleaning that apartment will not be for the faint of heart-it will take
someone with great courage. I think I'll call in the Marines to take care of it.   Adam is one of the few men who is willing to sign up to
protect our country. Unfortunately, they changed his specialty from artillery to infantry.  He was telling me that he now will carry a big gun
and he will be protecting the convoys. That is something a mother doesn't want to hear.  He is scheduled to go to Iraq in August of 2005.
\underline{\textbf{That is definitely something a mother doesn't want to hear}}.

Jasmine was the Maid of Honor.  Jazz was a Maid of Honor who managed to delegate all of her responsibilities to another bridesmaid. However, she
did show up for the wedding and she was beautiful. Jasmine is in her first year at California State University, Northridge, and is working
towards a degree in Graphic Arts.   She is truly an artist in every sense of the word.  Jasmine was working at a Subway restaurant and she has a
bit of a problem getting to work early or even getting there on time.  One afternoon, she flew out the door and not 10 minutes later, she called
and said she forgot her shoes.  How do you forget your shoes?  When you are dressed in a work uniform shouldn't that include shoes? Don't you
need shoes to drive?  She didn't have time for me to ask questions, she merely wanted me to find her shoes and when she honked I should bring
the shoes outside so she didn't have to stop.   On her return flyby, she slowed down and I tossed the shoes through the window and she was off
to work.   However, she does realize that she needs to multi-task in order to get to her early 11 a.m. class on time. The other day Nasr and I
were frantically searching for OUR toothpaste in OUR bathroom.  Finally, I went out and bought some.  The next morning I found the toothpaste in
the shower where Jazz had been multi-tasking.  Do you brush your teeth in the shower?   Jazz has her ears pierced in several places and when we
came back from the Caribbean, she was sporting a nose ring.  My first nasty thought was I hoped she got a really bad runny nose so she would
take it out. She never got a runny nose, but I noticed it was gone a few weeks ago.  So, yesterday, Thanksgiving Day, I was busy running around
trying to make dinner for 20 people and asking her for help.  She hemmed and hawed and escaped the kitchen.  Then suddenly, as I was taking the
stuffing out of the oven she matter of factly informed me that she has a belly button ring and she has had her belly pierced since she was 16!
She's proud that she kept it a secret for 2 years and she was telling me now because she is 18 and there is nothing I can do about it.   It's
true, she is 18, but that has nothing to do with the fact that I did not murder her on the spot.  The 20 HUNGRY witnesses saved her life.

I hope that the New Year doesn't bring any piercing surprises for anyone-even if they are of age.  This year, as every year, I want to wish
everyone a Happy, Healthy and Peaceful New Year.  However, this year, more than any year, praying for Peace on Earth is more important than ever
before.


